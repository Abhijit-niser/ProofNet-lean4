\documentclass{article}

\title{\textbf{
Exercises from \\
\textit{Real Mathematical Analysis} \\
by Charles Pugh
}}

\date{}

\usepackage{amsmath}
\usepackage{amssymb}
\usepackage{amsthm}

\begin{document}
\maketitle



\paragraph{Exercise 2.12a} Let $(p_n)$ be a sequence and $f:\mathbb{N}\to\mathbb{N}$. The sequence $(q_k)_{k\in\mathbb{N}}$ with $q_k=p_{f(k)}$ is called a rearrangement of $(p_n)$. Show that if $f$ is an injection, the limit of a sequence is unaffected by rearrangement.
\begin{proof}
    Let $\varepsilon>0$. Since $p_n \rightarrow L$, we have that, for all $n$ except $n \leq N$, $d\left(p_n, L\right)<\epsilon$. Let $S=\{n \mid f(n) \leq N\}$, let $n_0$ be the largest $n \in S$, we know there is such a largest $n$ because $f(n)$ is injective. Now we have that $\forall n>n_0 f(n)>N$ which implies that $p_{f(n)} \rightarrow L$, as required.
\end{proof}


\paragraph{Exercise 2.26} Prove that a set $U \subset M$ is open if and only if none of its points are limits of its complement.
\begin{proof}
    Assume that none of the points of $U$ are limits of its complement, and let us prove that $U$ is open. Assume by contradiction that $U$ is not open, so there exists $p \in M$ so that $\forall r>0$ there exists $q \in M$ with $d(p, q)<r$ but $q \notin U$. Applying this to $r=1 / n$ we obtain $q_n \in U^c$ such that $d\left(q_n, p\right)<1 / n$. But then $q_n \rightarrow p$ and $p$ is a limit of a sequence of points in $U^c$, a contradiction.

Assume now that $U$ is open. Assume by contradiction there exists $p \in U$ and $p_n \in U^c$ such that $p_n \rightarrow p$. Since $U$ is open, there exists $r>0$ such that $d(p, x)<r$ for $x \in M$ implies $x \in U$. But since $p_n \rightarrow p$, there exists $n_0 \in \mathbb{N}$ such that $n \geq n_0$ implies $d\left(p_n, p\right)<r$, therefore $p_n \in U$ for $n \geq n_0$, a contradiction since $p_n \in U^c$.
\end{proof}



\paragraph{Exercise 2.29} Let $\mathcal{T}$ be the collection of open subsets of a metric space $\mathrm{M}$, and $\mathcal{K}$ the collection of closed subsets. Show that there is a bijection from $\mathcal{T}$ onto $\mathcal{K}$.
\begin{proof}
   The bijection given by $x\mapsto X^C$ suffices.  
\end{proof}



\paragraph{Exercise 2.32a} Show that every subset of $\mathbb{N}$ is clopen.
\begin{proof}
    32. The one-point set $\{n\} \subset \mathbb{N}$ is open, since it contains all $m \in \mathbb{N}$ that satisfy $d(m, n)<\frac{1}{2}$. Every subset of $\mathbb{N}$ is a union of one-point sets, hence is open. Then every set it closed, since its complement is necessarily open.
\end{proof}



\paragraph{Exercise 2.41} Let $\|\cdot\|$ be any norm on $\mathbb{R}^{m}$ and let $B=\left\{x \in \mathbb{R}^{m}:\|x\| \leq 1\right\}$. Prove that $B$ is compact.
\begin{proof}
    Let us call $\|\cdot\|_E$ the Euclidean norm in $\mathbb{R}^m$. We start by claiming that there exist constants $C_1, C_2>0$ such that
$$
C_1\|x\|_E \leq\|x\| \leq C_2\|x\|_E, \forall x \in \mathbb{R}^m .
$$
Assuming (1) to be true, let us finish the problem. First let us show that $B$ is bounded w.r.t. $d_E$, which is how we call the Euclidean distance in $\mathbb{R}^m$. Indeed, given $x \in B,\|x\|_E \leq \frac{1}{C_1}\|x\| \leq \frac{1}{C_1}$. Hence $B \subset\left\{x \in \mathbb{R}^m: d_E(x, 0)<\frac{1}{C_1}+1\right\}$, which means $B$ is bounded w.r.t $d_E$.
Now let us show that $B$ is closed w.r.t. $d_E$. Let $x_n \rightarrow x$ w.r.t. $d_E$, where $x_n \in B$. Notice that this implies that $x_n \rightarrow x$ w.r.t. $d(x, y)=\|x-y\|$, the distance coming from $\|\cdot\|$, since by (1) we have
$$
d\left(x_n, x\right)=\left\|x_n-x\right\| \leq C_2\left\|x_n-x\right\|_E \rightarrow 0 .
$$
Also, notice that
$$
\|x\| \leq\left\|x_n-x\right\|+\left\|x_n\right\| \leq\left\|x_n-x\right\|+1,
$$
hence passing to the limit we obtain that $\|x\| \leq 1$, therefore $x \in B$ and so $B$ is closed w.r.t. $d_E$. Since $B$ is closed and bounded w.r.t. $d_E$, it must be compact. Now we claim that the identity function, $i d:\left(\mathbb{R}^m, d_E\right) \rightarrow\left(\mathbb{R}^m, d\right)$ where $\left(\mathbb{R}^m, d_E\right)$ means we are using the distance $d_E$ in $\mathbb{R}^m$ and $\left(\mathbb{R}^m, d\right)$ means we are using the distance $d$ in $\mathbb{R}^m$, is a homeomorphism. This follows by (1), since $i d$ is always a bijection, and it is continuous and its inverse is continuous by (1) (if $x_n \rightarrow x$ w.r.t. $d_E$, then $x_n \rightarrow x$ w.r.t. $d$ and vice-versa, by (1)). By a result we saw in class, since $B$ is compact in $\left(\mathbb{R}^m, d_E\right)$ and $i d$ is a homeomorphism, then $i d(B)=B$ is compact w.r.t. $d$.

We are left with proving (1). Notice that it suffices to prove that $C_1 \leq\|x\| \leq$ $C_2, \forall x \in \mathbb{R}^m$ with $\|x\|_E=1$. Indeed, if this is true, given $x \in \mathbb{R}^m$, either $\|x\|_E=0$ (which implies $x=0$ and (1) holds in this case), or $x /\|x\|_E=y$ is such that $\|y\|_E=1$, so $C_1 \leq\|y\| \leq C_2$, which implies $C_1\|x\|_E \leq\|x\| \leq C_2\|x\|_E$.
We want to show now that $\|\cdot\|$ is continuous w.r.t. $d_E$, that is, given $\varepsilon>0$ and $x \in \mathbb{R}^m$, there exists $\delta>0$ such that if $d_E(x, y)<\delta$, then $\|\mid x\|-\|y\| \|<\varepsilon$.

By the triangle inequality, $\|x\|-\|y\| \leq\|x-y\|$, and $\|y\|-\|x\| \leq\|x-y\|$, therefore
$$
|\|x||-\| y|\|\leq\| x-y \| .
$$
Writing now $x=\sum_{i=1}^m a_i e_i, y=\sum_{i=1}^m b_i e_i$, where $e_i=(0, \ldots, 1,0, \ldots, 0)$ (with 1 in the i-th component), we obtain by the triangle inequality,
$$
\begin{aligned}
\|x-y\| & =\left\|\sum_{i=1}^m\left(a_i-b_i\right) e_i\right\| \leq \sum_{i=1}^m\left|a_i-b_i\left\|\left|\left\|e_i\right\| \leq \max _{i=1, \ldots, m}\left\|e_i\right\| \sum_{i=1}^m\right| a_i-b_i \mid\right.\right. \\
& =\max _{i=1, \ldots, m}\left\|e_i\right\| d_{s u m}(x, y) \leq \max _{i=1, \ldots, m}\left\|e_i\right\| m d_{\max }(x, y) \\
& \leq \max _{i=1, \ldots, m}\left\|e_i\right\| m d_E(x, y) .
\end{aligned}
$$
Let $\delta=\frac{\varepsilon}{m \max _{i=1, \ldots, m}\left\|e_i\right\|}$. Then if $d_E(x, y)<\delta,\|x\|-\|y\|||<\varepsilon$.
Since $\|\cdot\|$ is continuous w.r.t. $d_E$ and $K=\left\{x \in \mathbb{R}^m:\|x\|_E=1\right\}$ is compact w.r.t. $d_E$, then the function $\|\cdot\|$ achieves a maximum and a minimum value on $K$. Call $C_1=\min _{x \in K}\|x\|, C_2=\max _{x \in K}\|x\|$. Then
$$
C_1 \leq\|x\| \leq C_2, \forall x \in \mathbb{R}^m \text { such that }\|x\|_E=1,
$$
which is what we needed.
\end{proof}



\paragraph{Exercise 2.46} Assume that $A, B$ are compact, disjoint, nonempty subsets of $M$. Prove that there are $a_0 \in A$ and $b_0 \in B$ such that for all $a \in A$ and $b \in B$ we have $d(a_0, b_0) \leq d(a, b)$.
\begin{proof}
Let $A$ and $B$ be compact, disjoint and non-empty subsets of $M$. We want to show that there exist $a_0 \in A, b_0 \in B$ such that for all $a \in A, b \in B$,
$$
d\left(a_0, b_0\right) \leq d(a, b) .
$$
We saw in class that the distance function $d: M \times M \rightarrow \mathbb{R}$ is continuous. We also saw in class that any continuous, real-valued function assumes maximum and minimum values on a compact set. Since $A$ and $B$ are compact, $A \times B$ is (non-empty) compact in $M \times M$. Therefore there exists $\left(a_0, b_0\right) \in A \times B$ such that $d\left(a_0, b_0\right) \leq d(a, b), \forall(a, b) \in A \times B$.
\end{proof}



\paragraph{Exercise 2.57} Show that if $S$ is connected, it is not true in general that its interior is connected.
\begin{proof}
    Consider $X=\mathbb{R}^2$ and
$$
A=([-2,0] \times[-2,0]) \cup([0,2] \times[0,2])
$$
which is connected, while $\operatorname{int}(A)$ is not connected.
To see this consider the continuous function $f: \mathbb{R}^2 \rightarrow \mathbb{R}$ is defined by $f(x, y)=x+y$. Let $U=f^{-1}(0,+\infty)$ which is open in $\mathbb{R}^2$ and so $U \cap \operatorname{int}(A)$ is open in $\operatorname{int}(A)$. Also, since $(0,0) \notin \operatorname{int}(A)$, so for all $(x, y) \in \operatorname{int}(A), f(x, y) \neq 0$ and $U \cap \operatorname{int}(A)=f^{-1}[0,+\infty) \cap \operatorname{int}(A)$ is closed in $\operatorname{int}(A)$. Furthermore, $(1,1)=f^{-1}(2) \in U \cap \operatorname{int}(A)$ shows that $U \cap \operatorname{int}(A) \neq \emptyset$ while $(-1,-1) \in \operatorname{int}(A)$ and $(-1,-1) \notin U$ shows that $U \cap \operatorname{int}(A) \neq \operatorname{int}(A)$.
\end{proof}


\paragraph{Exercise 2.92} Give a direct proof that the nested decreasing intersection of nonempty covering compact sets is nonempty.
\begin{proof}
    Let
$$
A_1 \supset A_2 \supset \cdots \supset A_n \supset \cdots
$$
be a nested decreasing sequence of compacts. Suppose that $\bigcap A_n=\emptyset$. Take $U_n=A_n^c$, then
$$
\bigcup U_n=\bigcup A_n^c=\left(\bigcap A_n\right)^c=A_1 .
$$
Here, I'm thinking of $A_1$ as the main metric space. Since $\left\{U_n\right\}$ is an open covering of $A_1$, we can extract a finite subcovering, that is,
$$
A_{\alpha_1}^c \cup A_{\alpha_2}^c \cup \cdots \cup A_{\alpha_m}^c \supset A_1
$$
or
$$
\left(A_1 \backslash A_{\alpha_1}\right) \cup\left(A_1 \backslash A_{\alpha_2}\right) \cup \cdots \cup\left(A_1 \backslash A_{\alpha_m}\right) \supset A_1 .
$$
But, this is true only if $A_{\alpha_i}=\emptyset$ for some $i$, a contradiction.
\end{proof}

\paragraph{Exercise 2.126} Suppose that $E$ is an uncountable subset of $\mathbb{R}$. Prove that there exists a point $p \in \mathbb{R}$ at which $E$ condenses.
\begin{proof}
    I think this is the proof by contrapositive that you were getting at.
Suppose that $E$ has no limit points at all. Pick an arbitrary point $x \in E$. Then $x$ cannot be a limit point, so there must be some $\delta>0$ such that the ball of radius $\delta$ around $x$ contains no other points of $E$ :
$$
B_\delta(x) \cap E=\{x\}
$$
Call this "point 1 ". For the next point, take the closest element to $x$ and on its left; that is, choose the point
$$
\max [E \cap(-\infty, x)]
$$
if it exists (that is important - if not, skip to the next step). Note that by the argument above, this supremum, should it exist, cannot equal $x$ and is therefore a new point in $E$.

Call this "point 2 ". Now take the first point to the right of $x$ for "point 3 ". Take the first point to the left of point 2 for "point 4 ". And so on, ad infinitum.

This gives a countable list of unique points; we must show that it exhausts the entire set $E$. Suppose not. Suppose there is some element $a<x$ which is never included in the list (picking $a$ on the negative side of $x$ is arbitrary, and the same argument would work for the second case). Then the element closest and to the right of $a$ in $E$ (which exists, by the no-limit-points argument at the beginning) is also not in the list; if it was, $a$ would have been in one of the next two spots. And same with that point (call it $a_1$ ); there is a closest $a_2>a_1 \in E$ such that $a_2$ is not in the list. Repeating, we generate an infinite monotone-increasing sequence $\left\{a_i\right\}$ of elements in $E$ and not in the list, which is clearly bounded above by $x$. By the Monotone
Convergence Theorem this sequence has a limit. But that means the sequence $\left\{a_i\right\} \subset E$ converges to a limit, and hence $E$ has a limit point, contradicting the assumption. Therefore our list exhausts $E$, and we have enumerated all its elements.
\end{proof}


\paragraph{Exercise 3.1} Assume that $f \colon \mathbb{R} \rightarrow \mathbb{R}$ satisfies $|f(t)-f(x)| \leq|t-x|^{2}$ for all $t, x$. Prove that $f$ is constant.
\begin{proof}
    We have $|f(t)-f(x)| \leq|t-x|^2, \forall t, x \in \mathbb{R}$. Fix $x \in \mathbb{R}$ and let $t \neq x$. Then
$$
\left|\frac{f(t)-f(x)}{t-x}\right| \leq|t-x| \text {, hence } \lim _{t \rightarrow x}\left|\frac{f(t)-f(x)}{t-x}\right|=0 \text {, }
$$
so $f$ is differentiable in $\mathbb{R}$ and $f^{\prime}=0$. This implies that $f$ is constant, as seen in class.
\end{proof}



\paragraph{Exercise 3.4} Prove that $\sqrt{n+1}-\sqrt{n} \rightarrow 0$ as $n \rightarrow \infty$.
\begin{proof}
    $$
\sqrt{n+1}-\sqrt{n}=\frac{(\sqrt{n+1}-\sqrt{n})(\sqrt{n+1}+\sqrt{n})}{\sqrt{n+1}+\sqrt{n}}=\frac{1}{\sqrt{n+1}+\sqrt{n}}<\frac{1}{2 \sqrt{n}}
$$
\end{proof}


\paragraph{Exercise 3.63a} Prove that $\sum 1/k(\log(k))^p$ converges when $p > 1$.
\begin{proof}
    Using the integral test, for a set $a$, we see
$$
\lim _{b \rightarrow \infty} \int_a^b \frac{1}{x \log (x)^c} d x=\lim _{b \rightarrow \infty}\left(\frac{\log (b)^{1-c}}{1-c}-\frac{\log (a)^{1-c}}{1-c}\right)
$$
which goes to infinity if $c \leq 1$ and converges if $c>1$. Thus,
$$
\sum_{n=2}^{\infty} \frac{1}{n \log (n)^c}
$$
converges if and only if $c>1$. 
\end{proof}



\paragraph{Exercise 3.63b} Prove that $\sum 1/k(\log(k))^p$ diverges when $p \leq 1$.
\begin{proof} 
    Using the integral test, for a set $a$, we see
$$
\lim _{b \rightarrow \infty} \int_a^b \frac{1}{x \log (x)^c} d x=\lim _{b \rightarrow \infty}\left(\frac{\log (b)^{1-c}}{1-c}-\frac{\log (a)^{1-c}}{1-c}\right)
$$
which goes to infinity if $c \leq 1$ and converges if $c>1$. Thus,
$$
\sum_{n=2}^{\infty} \frac{1}{n \log (n)^c}
$$
converges if and only if $c>1$. 
\end{proof}



\paragraph{Exercise 4.15a} A continuous, strictly increasing function $\mu \colon (0, \infty) \rightarrow (0, \infty)$ is a modulus of continuity if $\mu(s) \rightarrow 0$ as $s \rightarrow 0$. A function $f \colon [a, b] \rightarrow \mathbb{R}$ has modulus of continuity $\mu$ if $|f(s) - f(t)| \leq \mu(|s - t|)$ for all $s, t \in [a, b]$. Prove that a function is uniformly continuous if and only if it has a modulus of continuity.
\begin{proof}
    Suppose there exists a modulus of continuity $w$ for $f$, then fix $\varepsilon>0$, since $\lim _{s \rightarrow 0} w(s)=0$, there exists $\delta>0$ such that for any $|s|<\delta$, we have $w(s)<\varepsilon$, then we have for any $x, z \in X$ such that $d_X(x, z)<\delta$, we have $d_Y(f(x), f(z)) \leq w\left(d_X(x, z)\right)<\varepsilon$, which means $f$ is uniformly continuous.

Suppose $f:\left(X, d_X\right) \rightarrow\left(Y, d_Y\right)$ is uniformly continuous.
Let $\delta_1>0$ be such that $d_X(a, b)<\delta_1$ implies $d_Y(f(a), f(b))<1$.
Define $w:[0, \infty) \rightarrow[0, \infty]$ by
$$
w(s)= \begin{cases}\left.\sup \left\{d_Y(f(a), f(b))\right\} \mid d_X(a, b) \leq s\right\} & \text { if } s \leq \delta_1 \\ \infty & \text { if } s>\delta_1\end{cases}
$$
We'll show that $w$ is a modulus of continuity for $f \ldots$
By definition of $w$, it's immediate that $w(0)=0$ and it's clear that
$$
d_Y(f(a), f(b)) \leq w\left(d_X(a, b)\right)
$$
for all $a, b \in X$.
It remains to show $\lim _{s \rightarrow 0^{+}} w(s)=0$.
It's easily seen that $w$ is nonnegative and non-decreasing, hence $\lim _{s \rightarrow 0^{+}}=L$ for some $L \geq 0$, where $L=\inf w((0, \infty))$
Let $\epsilon>0$.
By uniform continuity of $f$, there exists $\delta>0$ such that $d_X(a, b)<\delta$ implies $d_Y(f(a), f(b))<\epsilon$, hence by definition of $w$, we get $w(\delta) \leq \epsilon$.
Thus $L \leq \epsilon$ for all $\epsilon>0$, hence $L=0$.
This completes the proof.
\end{proof}
\end{document}
