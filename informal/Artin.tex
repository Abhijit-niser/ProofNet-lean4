\documentclass{article}

\title{\textbf{
Exercises from \\
\textit{Algebra} \\
by Michael Artin
}}

\date{}

\usepackage{amsmath}
\usepackage{amssymb}
\usepackage{amsthm}

\begin{document}
\maketitle


\paragraph{Exercise 2.2.9} Let $H$ be the subgroup generated by two elements $a, b$ of a group $G$. Prove that if $a b=b a$, then $H$ is an abelian group.
\begin{proof}
    Since $a$ and $b$ commute, for any $g, h\in H$ we can write $g=a^ib^j$ and $h = a^kb^l$. Then $gh = a^ib^ja^kb^l = a^kb^la^ib^j = hg$. Thus $H$ is abelian. 
\end{proof}



\paragraph{Exercise 2.3.2} Prove that the products $a b$ and $b a$ are conjugate elements in a group.
\begin{proof}
    We have that $(a^{-1})ab(a^{-1})^{-1} = ba$. 
\end{proof}



\paragraph{Exercise 2.4.19} Prove that if a group contains exactly one element of order 2 , then that element is in the center of the group.
\begin{proof}
   Let $x$ be the element of order two. Consider the element $z=y^{-1} x y$, we have: $z^2=\left(y^{-1} x y\right)^2=\left(y^{-1} x y\right)\left(y^{-1} x y\right)=e$. So: $z=x$, and $y^{-1} x y=x$. So: $x y=y x$. So: $x$ is in the center of $G$. 
\end{proof}



\paragraph{Exercise 2.8.6} Prove that the center of the product of two groups is the product of their centers.
\begin{proof}
    We have that $(g_1, g_2)\cdot (h_1, h_2) = (h_1, h_2)\cdot (g_1, g_2)$ if and only if $g_1h_1 = h_1g_1$ and $g_2h_2 = h_2g_2$. 
\end{proof}



\paragraph{Exercise 2.11.3} Prove that a group of even order contains an element of order $2 .$
\begin{proof}
    Pair up if possible each element of $G$ with its inverse, and observe that
$$
g^2 \neq e \Longleftrightarrow g \neq g^{-1} \Longleftrightarrow \text { there exists the pair }\left(g, g^{-1}\right)
$$
Now, there is one element that has no pairing: the unit $e$ (since indeed $e=e^{-1} \Longleftrightarrow e^2=e$ ), so since the number of elements of $G$ is even there must be at least one element more, say $e \neq a \in G$, without a pairing, and thus $a=a^{-1} \Longleftrightarrow a^2=e$
\end{proof}



\paragraph{Exercise 3.2.7} Prove that every homomorphism of fields is injective.
\begin{proof}
    Suppose $f(a)=f(b)$, then $f(a-b)=0=f(0)$. If $u=(a-b) \neq 0$, then $f(u) f\left(u^{-1}\right)=f(1)=1$, but that means that $0 f\left(u^{-1}\right)=1$, which is impossible. Hence $a-b=0$ and $a=b$.
\end{proof}



\paragraph{Exercise 3.5.6} Let $V$ be a vector space which is spanned by a countably infinite set. Prove that every linearly independent subset of $V$ is finite or countably infinite.
\begin{proof}
    Let $A$ be the countable generating set, and let $U$ be an uncountable linearly independent set. It can be extended to a basis $B$ of the whole space. Now consider the subset $C$ of elements of $B$ that appear in the $B$-decompositions of elements of $A$.
Since only finitely many elements are involved in the decomposition of each element of $A$, the set $C$ is countable. But $C$ also clearly generates the vector space $V$. This contradicts the fact that it is a proper subset of the basis $B$ (since $B$ is uncountable).
\end{proof}



\paragraph{Exercise 3.7.2} Let $V$ be a vector space over an infinite field $F$. Prove that $V$ is not the union of finitely many proper subspaces.
\begin{proof}
    If $V$ is the set-theoretic union of $n$ proper subspaces $W_i$ ( $1 \leq i \leq n$ ), then $|F| \leq n-1$.
Proof. We may suppose no $W_i$ is contained in the union of the other subspaces. Let $u \in W_i, \quad u \notin \bigcup_{j \neq i} W_j$ and $v \notin W_i$.
Then $(v+F u) \cap W_i=\varnothing$ and $(v+F u) \cap W_j(j \neq i)$ contains at most one vector since otherwise $W_j$ would contain $u$. Hence
$$
|v+F u|=|F| \leq n-1 .
$$
Corollary: Avoidance lemma for vector spaces.
Let $E$ be a vector space over an infinite field. If a subspace is contained in a finite union of subspaces, it is contained in one of them.
\end{proof}



\paragraph{Exercise 6.1.14} Let $Z$ be the center of a group $G$. Prove that if $G / Z$ is a cyclic group, then $G$ is abelian and hence $G=Z$.
\begin{proof}
    We have that $G / Z(G)$ is cyclic, and so there is an element $x \in G$ such that $G / Z(G)=\langle x Z(G)\rangle$, where $x Z(G)$ is the coset with representative $x$. Now let $g \in G$
We know that $g Z(G)=(x Z(G))^m$ for some $m$, and by definition $(x Z(G))^m=x^m Z(G)$.
Now, in general, if $H \leq G$, we have by definition too that $a H=b H$ if and only if $b^{-1} a \in H$.
In our case, we have that $g Z(G)=x^m Z(G)$, and this happens if and only if $\left(x^m\right)^{-1} g \in Z(G)$.
Then, there's a $z \in Z(G)$ such that $\left(x^m\right)^{-1} g=z$, and so $g=x^m z$.

$g, h \in G$ implies that $g=x^{a_1} z_1$ and $h=x^{a_2} z_2$, so
$$
\begin{aligned}
g h & =\left(x^{a_1} z_1\right)\left(x^{a_2} z_2\right) \\
& =x^{a_1} x^{a_2} z_1 z_2 \\
& =x^{a_1+a_2} z_2 z_1 \\
& =\ldots=\left(x^{a_2} z_2\right)\left(x^{a_1} z_1\right)=h g .
\end{aligned}
$$
Therefore, $G$ is abelian.
\end{proof}



\paragraph{Exercise 6.4.2} Prove that no group of order $p q$, where $p$ and $q$ are prime, is simple.
\begin{proof}
    If $|G|=n=p q$ then the only two Sylow subgroups are of order $p$ and $q$.
From Sylow's third theorem we know that $n_p \mid q$ which means that $n_p=1$ or $n_p=q$.
If $n_p=1$ then we are done (by a corollary of Sylow's theorem)
If $n_p=q$ then we have accounted for $q(p-1)=p q-q$ elements of $G$ and so there is only one group of order $q$ and again we are done.
\end{proof}



\paragraph{Exercise 6.4.3} Prove that no group of order $p^2 q$, where $p$ and $q$ are prime, is simple.
\begin{proof}
    We may as well assume $p<q$. The number of Sylow $q$-subgroups is $1 \bmod q$ and divides $p^2$. So it is $1, p$, or $p^2$. We win if it's 1 and it can't be $p$, so suppose it's $p^2$. But now $q \mid p^2-1$, so $q \mid p+1$ or $q \mid p-1$.
Thus $p=2$ and $q=3$. But we know no group of order 36 is simple. 
\end{proof}



\paragraph{Exercise 6.4.12} Prove that no group of order 224 is simple.
\begin{proof}
    The following proves there must exist a normal Sylow 2 -subgroup of order 32 ,
Suppose there are $n_2=7$ Sylow 2 -subgroups in $G$. Making $G$ act on the set of these Sylow subgroups by conjugation (Mitt wrote about this but on the set of the other Sylow subgroups, which gives no contradiction), we get a homomorphism $G \rightarrow S_7$ which must be injective if $G$ is simple (why?).

But this cannot be since then we would embed $G$ into $S_7$, which is impossible since $|G| \nmid 7 !=\left|S_7\right|$ (why?)
\end{proof}



\paragraph{Exercise 6.8.1} Prove that two elements $a, b$ of a group generate the same subgroup as $b a b^2, b a b^3$.
\begin{proof}
    Let $H = \langle bab^2, bab^3\rangle$. It is clear that $H\subset \langle a, b\rangle$. Note that $(bab^2)^{-1}(bab^3)=b$, therefore $b\in H$. This then implies that $b^{-1}(bab^2)b^{-2}=a\in H$. Thus $\langle a, b\rangle\subset H$.  
\end{proof}

\paragraph{Exercise 10.1.13} An element $x$ of a ring $R$ is called nilpotent if some power of $x$ is zero. Prove that if $x$ is nilpotent, then $1+x$ is a unit in $R$.
\begin{proof}
    If $x^n=0$, then
$$
(1+x)\left(\sum_{k=0}^{n-1}(-1)^k x^k\right)=1+(-1)^{n-1} x^n=1 .
$$
\end{proof}



\paragraph{Exercise 10.2.4} Prove that in the ring $\mathbb{Z}[x],(2) \cap(x)=(2 x)$.
\begin{proof}
    Let $f(x) \in(2 x)$. Then there exists some polynomial $g(x) \in \mathbb{Z}$ such that
$$
f(x)=2 x g(x)
$$
But this means that $f(x) \in(2)$ (because $x g(x)$ is a polynomial), and $f(x) \in$ $(x)$ (because $2 g(x)$ is a polynomial). Thus, $f(x) \in(2) \cap(x)$, and
$$
(2 x) \subseteq(2) \cap(x)
$$
On the other hand, let $p(x) \in(2) \cap(x)$. Since $p(x) \in(2)$, there exists some polynomial $h(x) \in \mathbb{Z}[x]$ such that
$$
p(x)=2 h(x)
$$
Furthermore, $p(x) \in(x)$, so
$$
p(x)=x h_2(x)
$$
So, $2 h(x)=x h_2(x)$, for some $h_2(x) \in \mathbb{Z}[x]$. This means that $h(0)=0$, so $x$ divides $h(x)$; that is,
$$
h(x)=x q(x)
$$
for some $q(x) \in \mathbb{Z}[x]$, and
$$
p(x)=2 x q(x)
$$
Thus, $p(x) \in(2 x)$, and
$$
\text { (2) } \cap(x) \subseteq(2 x)
$$
Finally,
(2) $\cap(x)=(2 x)$,
as required.
\end{proof}



\paragraph{Exercise 10.6.7} Prove that every nonzero ideal in the ring of Gauss integers contains a nonzero integer.
\begin{proof}
    Let $I$ be some nonzero ideal. Then there exists some $z \in \mathbb{Z}[i], z \neq 0$ such that $z \in I$. We know that $z=a+b i$, for some $a, b \in \mathbb{Z}$. We consider three cases:
1. If $b=0$, then $z=a$, so $z \in \mathbb{Z} \cap I$, and $z \neq 0$, so the statement of the exercise holds.
2. If $a=0$, then $z=i b$. Since $z \neq 0$, we conclude that $b \neq 0$. Since $I$ is an ideal in $\mathbb{Z}[i]$, and $i \in \mathbb{Z}[i]$, we conclude that $i z \in I$. Furthermore, $i z=-b \in \mathbb{Z}$. Thus, $i z$ is a nonzero integer which is in $I$.
3. Let $a \neq 0$ and $b \neq 0$. Since $I$ is an ideal and $z \in I$, we conclude that $z^2 \in I$; that is,
$$
(a+b i)^2=a^2-b^2+2 a b i \in I
$$
Furthermore, since $-2 a \in \mathbb{Z}[i]$, and $z \in I$ and $I$ is an ideal, $-2 a z \in I$; that is,
$$
-2 a z=-2 a(a+b i)=-2 a^2-2 a b i \in I
$$
Since $I$ is closed under addition,
$$
\left(a^2-b^2+2 a b i\right)+\left(-2 a^2-2 a b i\right) \in I \Longrightarrow-a^2-b^2 \in I
$$
Notice that $-a^2-b^2 \neq 0$ since $a^2>0$ and $b^2>0$, so $-a^2-b^2<0$. Furthermore, it is an integer. Thus, we have found a nonzero integer in $I$.
\end{proof}



\paragraph{Exercise 10.4.6} Let $I, J$ be ideals in a ring $R$. Prove that the residue of any element of $I \cap J$ in $R / I J$ is nilpotent.
\begin{proof}
    If $x$ is in $I \cap J, x \in I$ and $x \in J . R / I J=\{r+a b: a \in I, b \in J, r \in R\}$. Then $x \in I \cap J \Rightarrow x \in I$ and $x \in J$, and so $x^2 \in I J$. Thus
$$
[x]^2=\left[x^2\right]=[0] \text { in } R / I J
$$
\end{proof}



\paragraph{Exercise 10.4.7a} Let $I, J$ be ideals of a ring $R$ such that $I+J=R$. Prove that $I J=I \cap J$.
\begin{proof}
    We have seen that $IJ \subset I \cap J$, so it remains to show that $I \cap J \subset IJ$.  Since $I+J = (1)$, there are elements $i \in I$ and $j \in J$ such that $i+j = 1$.  Let $k \in I \cap J$, and multiply $i+j=1$ through by $k$ to get $ki+kj = k$.  Write this more suggestively as
\[ k = ik+kj. \]
The first term is in $IJ$ because $k \in J$, and the second term is in $IJ$ because $k \in I$, so $k \in IJ$ as desired.
\end{proof}

\paragraph{Exercise 10.7.10} Let $R$ be a ring, with $M$ an ideal of $R$. Suppose that every element of $R$ which is not in $M$ is a unit of $R$. Prove that $M$ is a maximal ideal and that moreover it is the only maximal ideal of $R$.
\begin{proof}
Suppose there is an ideal $M\subset I\subset R$. If $I\neq M$, then $I$ contains a unit, thus $I=R$. Therefore $M$ is a maximal ideal. 

Suppose we have an arbitrary maximal ideal $M^\prime$ of $R$. The ideal $M^\prime$ cannot contain a unit, otherwise $M^\prime =R$. Therefore $M^\prime \subset M$. But we cannot have $M^\prime \subsetneq M \subsetneq R$, therefore $M=M^\prime$. 
\end{proof}



\paragraph{Exercise 11.2.13} If $a, b$ are integers and if $a$ divides $b$ in the ring of Gauss integers, then $a$ divides $b$ in $\mathbb{Z}$.
\begin{proof}
    Suppose $a|b$ in $\mathbb{Z}[i]$ and $a,b\in\mathbb{Z}$. Then $a(x+yi)=b$ for $x,y\in\mathbb{Z}$. Expanding this we get $ax+ayi=b$, and equating imaginary parts gives us $ay=0$, implying $y=0$. 
\end{proof}


\paragraph{Exercise 11.4.1b} Prove that $x^3 + 6x + 12$ is irreducible in $\mathbb{Q}$.
\begin{proof}
    Apply Eisenstein's criterion with $p=3$. 
\end{proof}



\paragraph{Exercise 11.4.6a} Prove that $x^2+x+1$ is irreducible in the field $\mathbb{F}_2$.
\begin{proof}
    If $x^2+x+1$ were reducible in $\mathbb{F}_2$, its factors must be linear. But we neither have that $0^2+0+1=$ nor $1^2+1+1=0$, therefore $x^2+x+1$ is irreducible.  
\end{proof}


\paragraph{Exercise 11.4.6b} Prove that $x^2+1$ is irreducible in $\mathbb{F}_7$
\begin{proof}
    If $p(x)=x^2+1$ were reducible, its factors must be linear. But no $p(a)$ for $a\in\mathbb{F}_7$ evaluates to 0, therefore $x^2+1$ is irreducible. 
\end{proof}



\paragraph{Exercise 11.4.6c} Prove that $x^3 - 9$ is irreducible in $\mathbb{F}_{31}$.
\begin{proof}
    If $p(x) = x^3-9$ were reducible, it would have a linear factor, since it either has a linear factor and a quadratic factor or three linear factors. We can then verify by brute force that $p(x)\neq 0$ for $x \in \mathbb{F}_31$. 
\end{proof}



\paragraph{Exercise 11.4.8} Let $p$ be a prime integer. Prove that the polynomial $x^n-p$ is irreducible in $\mathbb{Q}[x]$.
\begin{proof}
   Straightforward application of Eisenstein's criterion with $p$.  
\end{proof}

\paragraph{Exercise 11.13.3} Prove that there are infinitely many primes congruent to $-1$ (modulo $4$).
\begin{proof}
    First we show a lemma: if $a \equiv 3(\bmod 4)$ then there exists a prime $p$ such that $p \mid a$ and $p \equiv 3(\bmod 4)$.

    Clearly, all primes dividing $a$ are odd. Suppose all of them would be $\equiv 1(\bmod 4)$. Then their product would also be $a \equiv 1(\bmod 4)$, which is a contradiction.

To prove the main claim, suppose that $p_1, \ldots, p_n$ would be all such primes. (In particular, we have $p_1=3$.) Consider $a=4 p_2 \cdots p_n+3$. (Or you can take $a=4 p_2 \cdots p_n-1$.) Show that $p_i \nmid a$ for $i=1, \ldots, n$. (The case $3 \nmid a$ is solved differently than the other primes - this is the reason for omitting $p_1$ in the definition of $a$.) Then use the above lemma to get a contradiction.
\end{proof}



\paragraph{Exercise 13.4.10} Prove that if a prime integer $p$ has the form $2^r+1$, then it actually has the form $2^{2^k}+1$.
\begin{proof}
    In particular, we have
$$
\frac{x^a+1}{x+1}=\frac{(-x)^a-1}{(-x)-1}=1-x+x^2-\cdots+(-x)^{a-1}
$$
by the geometric sum formula. In this case, specialize to $x=2^{2^m}$ and we have a nontrivial divisor.
\end{proof}



\paragraph{Exercise 13.6.10} Let $K$ be a finite field. Prove that the product of the nonzero elements of $K$ is $-1$.
\begin{proof}
    Since we are working with a finite field with $q$ elements, anyone of them is a root of the following polynomial
$$
x^q-x=0 .
$$
In particular if we rule out the 0 element, any $a_i \neq 0$ is a root of
$$
x^{q-1}-1=0 .
$$
This polynomial splits completely in $\mathbb{F}_q$ so we find
$$
\left(x-a_1\right) \cdots\left(x-a_{q-1}\right)=0
$$
in particular
$$
x^{q-1}-1=\left(x-a_1\right) \cdots\left(x-a_{q-1}\right)
$$
Thus $a_1 \cdots a_{q-1}=-1$.
\end{proof}
\end{document}
