\documentclass{article}

\title{\textbf{
Exercises from \\
\textit{Abstract Algebra} \\
by David Dummit and Richard Foote
}}

\date{}

\usepackage{amsmath, amsthm}
\usepackage{amssymb}


\theoremstyle{definition}
\newtheorem{lem}{Lemma}[section]
\newtheorem{cor}[lem]{Corollary}
\newtheorem{prop}[lem]{Proposition}
\newtheorem{thm}[lem]{Theorem}
\newtheorem{remark}[lem]{Remark}
\newtheorem{defn}[lem]{Definition}

\newtheorem*{prop*}{Proposition}
\newtheorem*{thm*}{Theorem}
\newtheorem*{defn*}{Definition}
\newtheorem*{lem*}{Lemma}

\begin{document}
\maketitle


\paragraph{Exercise 1.1.2a} Prove the the operation $\star$ on $\mathbb{Z}$ defined by $a\star b=a-b$ is not commutative.
\begin{proof}
    Not commutative since
$$
1 \star(-1)=1-(-1)=2
$$
$$
(-1) \star 1=-1-1=-2 .
$$
\end{proof}



\paragraph{Exercise 1.1.3} Prove that the addition of residue classes $\mathbb{Z}/n\mathbb{Z}$ is associative.
\begin{proof}
    We have
$$
\begin{aligned}
(\bar{a}+\bar{b})+\bar{c} &=\overline{a+b}+\bar{c} \\
&=\overline{(a+b)+c} \\
&=\overline{a+(b+c)} \\
&=\bar{a}+\overline{b+c} \\
&=\bar{a}+(\bar{b}+\bar{c})
\end{aligned}
$$
since integer addition is associative.
\end{proof}



\paragraph{Exercise 1.1.4} Prove that the multiplication of residue class $\mathbb{Z}/n\mathbb{Z}$ is associative.
\begin{proof}
    We have
$$
\begin{aligned}
(\bar{a} \cdot \bar{b}) \cdot \bar{c} &=\overline{a \cdot b} \cdot \bar{c} \\
&=\overline{(a \cdot b) \cdot c} \\
&=\overline{a \cdot(b \cdot c)} \\
&=\bar{a} \cdot \overline{b \cdot c} \\
&=\bar{a} \cdot(\bar{b} \cdot \bar{c})
\end{aligned}
$$
since integer multiplication is associative.
\end{proof}



\paragraph{Exercise 1.1.5} Prove that for all $n>1$ that $\mathbb{Z}/n\mathbb{Z}$ is not a group under multiplication of residue classes.
\begin{proof}
    Note that since $n>1, \overline{1} \neq \overline{0}$. Now suppose $\mathbb{Z} /(n)$ contains a multiplicative identity element $\bar{e}$. Then in particular,
$$
\bar{e} \cdot \overline{1}=\overline{1}
$$
so that $\bar{e}=\overline{1}$. Note, however, that
$$
\overline{0} \cdot \bar{k}=\overline{0}
$$
for all k, so that $\overline{0}$ does not have a multiplicative inverse. Hence $\mathbb{Z} /(n)$ is not a group under multiplication.
\end{proof}



\paragraph{Exercise 1.1.15} Prove that $(a_1a_2\dots a_n)^{-1} = a_n^{-1}a_{n-1}^{-1}\dots a_1^{-1}$ for all $a_1, a_2, \dots, a_n\in G$.
\begin{proof}
    For $n=1$, note that for all $a_1 \in G$ we have $a_1^{-1}=a_1^{-1}$.
Now for $n \geq 2$ we proceed by induction on $n$. For the base case, note that for all $a_1, a_2 \in G$ we have
$$
\left(a_1 \cdot a_2\right)^{-1}=a_2^{-1} \cdot a_1^{-1}
$$
since
$$
a_1 \cdot a_2 \cdot a_2^{-1} a_1^{-1}=1 .
$$
For the inductive step, suppose that for some $n \geq 2$, for all $a_i \in G$ we have
$$
\left(a_1 \cdot \ldots \cdot a_n\right)^{-1}=a_n^{-1} \cdot \ldots \cdot a_1^{-1} .
$$
Then given some $a_{n+1} \in G$, we have
$$
\begin{aligned}
\left(a_1 \cdot \ldots \cdot a_n \cdot a_{n+1}\right)^{-1} &=\left(\left(a_1 \cdot \ldots \cdot a_n\right) \cdot a_{n+1}\right)^{-1} \\
&=a_{n+1}^{-1} \cdot\left(a_1 \cdot \ldots \cdot a_n\right)^{-1} \\
&=a_{n+1}^{-1} \cdot a_n^{-1} \cdot \ldots \cdot a_1^{-1},
\end{aligned}
$$
using associativity and the base case where necessary.
\end{proof}



\paragraph{Exercise 1.1.16} Let $x$ be an element of $G$. Prove that $x^2=1$ if and only if $|x|$ is either $1$ or $2$.
\begin{proof}
    $(\Rightarrow)$ Suppose $x^2=1$. Then we have $0<|x| \leq 2$, i.e., $|x|$ is either 1 or 2 .
( $\Leftarrow$ ) If $|x|=1$, then we have $x=1$ so that $x^2=1$. If $|x|=2$ then $x^2=1$ by definition. So if $|x|$ is 1 or 2 , we have $x^2=1$.
\end{proof}


\paragraph{Exercise 1.1.17} Let $x$ be an element of $G$. Prove that if $|x|=n$ for some positive integer $n$ then $x^{-1}=x^{n-1}$.
\begin{proof}
    We have $x \cdot x^{n-1}=x^n=1$, so by the uniqueness of inverses $x^{-1}=x^{n-1}$.
\end{proof}



\paragraph{Exercise 1.1.18} Let $x$ and $y$ be elements of $G$. Prove that $xy=yx$ if and only if $y^{-1}xy=x$ if and only if $x^{-1}y^{-1}xy=1$.
\begin{proof}
If $x y=y x$, then $y^{-1} x y=y^{-1} y x=1 x=x$. Multiplying by $x^{-1}$ then gives $x^{-1} y^{-1} x y=1$.

On the other hand, if $x^{-1} y^{-1} x y=1$, then we may multiply on the left by $x$ to get $y^{-1} x y=x$. Then multiplying on the left by $y$ gives $x y=y x$ as desired.
\end{proof}




\paragraph{Exercise 1.1.20} For $x$ an element in $G$ show that $x$ and $x^{-1}$ have the same order.
\begin{proof}
    Recall that the order of a group element is either a positive integer or infinity.
Suppose $|x|$ is infinite and that $\left|x^{-1}\right|=n$ for some $n$. Then
$$
x^n=x^{(-1) \cdot n \cdot(-1)}=\left(\left(x^{-1}\right)^n\right)^{-1}=1^{-1}=1,
$$
a contradiction. So if $|x|$ is infinite, $\left|x^{-1}\right|$ must also be infinite. Likewise, if $\left|x^{-1}\right|$ is infinite, then $\left|\left(x^{-1}\right)^{-1}\right|=|x|$ is also infinite.
Suppose now that $|x|=n$ and $\left|x^{-1}\right|=m$ are both finite. Then we have
$$
\left(x^{-1}\right)^n=\left(x^n\right)^{-1}=1^{-1}=1,
$$
so that $m \leq n$. Likewise, $n \leq m$. Hence $m=n$ and $x$ and $x^{-1}$ have the same order.
\end{proof}


\paragraph{Exercise 1.1.22a} If $x$ and $g$ are elements of the group $G$, prove that $|x|=\left|g^{-1} x g\right|$.
\begin{proof}
    First we prove a technical lemma:

    {\bf Lemma.} For all $a, b \in G$ and $n \in \mathbb{Z},\left(b^{-1} a b\right)^n=b^{-1} a^n b$.
The statement is clear for $n=0$. We prove the case $n>0$ by induction; the base case $n=1$ is clear. Now suppose $\left(b^{-1} a b\right)^n=b^{-1} a^n b$ for some $n \geq 1$; then
$$
\left(b^{-1} a b\right)^{n+1}=\left(b^{-1} a b\right)\left(b^{-1} a b\right)^n=b^{-1} a b b^{-1} a^n b=b^{-1} a^{n+1} b .
$$
By induction the statement holds for all positive $n$.
Now suppose $n<0$; we have
$$
\left(b^{-1} a b\right)^n=\left(\left(b^{-1} a b\right)^{-n}\right)^{-1}=\left(b^{-1} a^{-n} b\right)^{-1}=b^{-1} a^n b .
$$
Hence, the statement holds for all integers $n$.
Now to the main result. Suppose first that $|x|$ is infinity and that $\left|g^{-1} x g\right|=n$ for some positive integer $n$. Then we have
$$
\left(g^{-1} x g\right)^n=g^{-1} x^n g=1,
$$
and multiplying on the left by $g$ and on the right by $g^{-1}$ gives us that $x^n=1$, a contradiction. Thus if $|x|$ is infinity, so is $\left|g^{-1} x g\right|$. Similarly, if $\left|g^{-1} x g\right|$ is infinite and $|x|=n$, we have
$$
\left(g^{-1} x g\right)^n=g^{-1} x^n g=g^{-1} g=1,
$$
a contradiction. Hence if $\left|g^{-1} x g\right|$ is infinite, so is $|x|$.
Suppose now that $|x|=n$ and $\left|g^{-1} x g\right|=m$ for some positive integers $n$ and $m$. We have
$$
\left(g^{-1} x g\right)^n=g^{-1} x^n g=g^{-1} g=1,
$$
So that $m \leq n$, and
$$
\left(g^{-1} x g\right)^m=g^{-1} x^m g=1,
$$
so that $x^m=1$ and $n \leq m$. Thus $n=m$.
\end{proof}



\paragraph{Exercise 1.1.22b} Deduce that $|a b|=|b a|$ for all $a, b \in G$.
\begin{proof}
    Let $a$ and $b$ be arbitrary group elements. Letting $x=a b$ and $g=a$, we see that
$$
|a b|=\left|a^{-1} a b a\right|=|b a| .
$$
\end{proof}



\paragraph{Exercise 1.1.25} Prove that if $x^{2}=1$ for all $x \in G$ then $G$ is abelian.
\begin{proof}
    Solution: Note that since $x^2=1$ for all $x \in G$, we have $x^{-1}=x$. Now let $a, b \in G$. We have
$$
a b=(a b)^{-1}=b^{-1} a^{-1}=b a .
$$
Thus $G$ is abelian.
\end{proof}



\paragraph{Exercise 1.1.29} Prove that $A \times B$ is an abelian group if and only if both $A$ and $B$ are abelian.
\begin{proof}
    $(\Rightarrow)$ Suppose $a_1, a_2 \in A$ and $b_1, b_2 \in B$. Then
$$
\left(a_1 a_2, b_1 b_2\right)=\left(a_1, b_1\right) \cdot\left(a_2, b_2\right)=\left(a_2, b_2\right) \cdot\left(a_1, b_1\right)=\left(a_2 a_1, b_2 b_1\right) .
$$
Since two pairs are equal precisely when their corresponding entries are equal, we have $a_1 a_2=a_2 a_1$ and $b_1 b_2=b_2 b_1$. Hence $A$ and $B$ are abelian.
$(\Leftarrow)$ Suppose $\left(a_1, b_1\right),\left(a_2, b_2\right) \in A \times B$. Then we have
$$
\left(a_1, b_1\right) \cdot\left(a_2, b_2\right)=\left(a_1 a_2, b_1 b_2\right)=\left(a_2 a_1, b_2 b_1\right)=\left(a_2, b_2\right) \cdot\left(a_1, b_1\right) .
$$
Hence $A \times B$ is abelian.
\end{proof}



\paragraph{Exercise 1.1.34} If $x$ is an element of infinite order in $G$, prove that the elements $x^{n}, n \in \mathbb{Z}$ are all distinct.
\begin{proof}
    Solution: Suppose to the contrary that $x^a=x^b$ for some $0 \leq a<b \leq n-1$. Then we have $x^{b-a}=1$, with $1 \leq b-a<n$. However, recall that $n$ is by definition the least integer $k$ such that $x^k=1$, so we have a contradiction. Thus all the $x^i$, $0 \leq i \leq n-1$, are distinct. In particular, we have
$$
\left\{x^i \mid 0 \leq i \leq n-1\right\} \subseteq G,
$$
so that $|x|=n \leq|G|$
\end{proof}



\paragraph{Exercise 1.3.8} Prove that if $\Omega=\{1,2,3, \ldots\}$ then $S_{\Omega}$ is an infinite group
\begin{proof}
    Recall that the codomain of an injective function must be at least as large (in cardinality) as the domain of the function. With that in mind, define the function
$$
\begin{gathered}
f: \mathbb{N} \rightarrow S_{\mathbb{N}} \\
f(n)=(1 n)
\end{gathered}
$$
where $(1 n)$ is the cycle decomposition of an element of $S_{\mathbb{N}}$ (specifically it's the function given by $g(1)=n, g(2)=2, g(3)=3, \ldots)$. The function $f$ maps every natural number to a distinct one of these functions. Hence $f$ is injective. Hence $\infty=|\mathbb{N}| \leq\left|S_{\mathbb{N}}\right|$.
\end{proof}



\paragraph{Exercise 1.6.4} Prove that the multiplicative groups $\mathbb{R}-\{0\}$ and $\mathbb{C}-\{0\}$ are not isomorphic.
\begin{proof}
    Isomorphic groups necessarily have the same number of elements of order $n$ for all finite $n$.

Now let $x \in \mathbb{R}^{\times}$. If $x=1$ then $|x|=1$, and if $x=-1$ then $|x|=2$. If (with bars denoting absolute value) $|x|<1$, then we have
$$
1>|x|>\left|x^2\right|>\cdots,
$$
and in particular, $1>\left|x^n\right|$ for all $n$. So $x$ has infinite order in $\mathbb{R}^{\times}$.
Similarly, if $|x|>1$ (absolute value) then $x$ has infinite order in $\mathbb{R}^{\times}$. So $\mathbb{R}^{\times}$has 1 element of order 1,1 element of order 2 , and all other elements have infinite order.
In $\mathbb{C}^{\times}$, on the other hand, $i$ has order 4 . Thus $\mathbb{R}^{\times}$and $\mathbb{C}^{\times}$are not isomorphic.
\end{proof}



\paragraph{Exercise 1.6.11} Let $A$ and $B$ be groups. Prove that $A \times B \cong B \times A$.
\begin{proof}
    We know from set theory that the mapping $\varphi: A \times B \rightarrow B \times A$ given by $\varphi((a, b))=(b, a)$ is a bijection with inverse $\psi: B \times A \rightarrow A \times B$ given by $\psi((b, a))=(a, b)$. Also $\varphi$ is a homomorphism, as we show below.
Let $a_1, a_2 \in A$ and $b_1, b_2 \in B$. Then
$$
\begin{aligned}
\varphi\left(\left(a_1, b_1\right) \cdot\left(a_2, b_2\right)\right) &=\varphi\left(\left(a_1 a_2, b_1 b_2\right)\right) \\
&=\left(b_1 b_2, a_1 a_2\right) \\
&=\left(b_1, a_1\right) \cdot\left(b_2, a_2\right) \\
&=\varphi\left(\left(a_1, b_1\right)\right) \cdot \varphi\left(\left(a_2, b_2\right)\right)
\end{aligned}
$$
Hence $A \times B \cong B \times A$.
\end{proof}



\paragraph{Exercise 1.6.17} Let $G$ be any group. Prove that the map from $G$ to itself defined by $g \mapsto g^{-1}$ is a homomorphism if and only if $G$ is abelian.
\begin{proof}
    $(\Rightarrow)$ Suppose $G$ is abelian. Then
$$
\varphi(a b)=(a b)^{-1}=b^{-1} a^{-1}=a^{-1} b^{-1}=\varphi(a) \varphi(b),
$$
so that $\varphi$ is a homomorphism.
$(\Leftarrow)$ Suppose $\varphi$ is a homomorphism, and let $a, b \in G$. Then
$$
a b=\left(b^{-1} a^{-1}\right)^{-1}=\varphi\left(b^{-1} a^{-1}\right)=\varphi\left(b^{-1}\right) \varphi\left(a^{-1}\right)=\left(b^{-1}\right)^{-1}\left(a^{-1}\right)^{-1}=b a,
$$
so that $G$ is abelian.
\end{proof}



\paragraph{Exercise 1.6.23} Let $G$ be a finite group which possesses an automorphism $\sigma$ such that $\sigma(g)=g$ if and only if $g=1$. If $\sigma^{2}$ is the identity map from $G$ to $G$, prove that $G$ is abelian.
\begin{proof}
    Solution: We define a mapping $f: G \rightarrow G$ by $f(x)=x^{-1} \sigma(x)$.
Claim: $f$ is injective.
Proof of claim: Suppose $f(x)=f(y)$. Then $y^{-1} \sigma(y)=x^{-1} \sigma(x)$, so that $x y^{-1}=\sigma(x) \sigma\left(y^{-1}\right)$, and $x y^{-1}=\sigma\left(x y^{-1}\right)$. Then we have $x y^{-1}=1$, hence $x=y$. So $f$ is injective.

Since $G$ is finite and $f$ is injective, $f$ is also surjective. Then every $z \in G$ is of the form $x^{-1} \sigma(x)$ for some $x$. Now let $z \in G$ with $z=x^{-1} \sigma(x)$. We have
$$
\sigma(z)=\sigma\left(x^{-1} \sigma(x)\right)=\sigma(x)^{-1} x=\left(x^{-1} \sigma(x)\right)^{-1}=z^{-1} .
$$
Thus $\sigma$ is in fact the inversion mapping, and we assumed that $\sigma$ is a homomorphism. By a previous example, then, $G$ is abelian.
\end{proof}



\paragraph{Exercise 2.1.5} Prove that $G$ cannot have a subgroup $H$ with $|H|=n-1$, where $n=|G|>2$.
\begin{proof}
    Solution: Under these conditions, there exists a nonidentity element $x \in H$ and an element $y \notin H$. Consider the product $x y$. If $x y \in H$, then since $x^{-1} \in H$ and $H$ is a subgroup, $y \in H$, a contradiction. If $x y \notin H$, then we have $x y=y$. Thus $x=1$, a contradiction. Thus no such subgroup exists.
\end{proof}



\paragraph{Exercise 2.1.13} Let $H$ be a subgroup of the additive group of rational numbers with the property that $1 / x \in H$ for every nonzero element $x$ of $H$. Prove that $H=0$ or $\mathbb{Q}$.
\begin{proof}
    Solution: First, suppose there does not exist a nonzero element in $H$. Then $H=0$.
Now suppose there does exist a nonzero element $a \in H$; without loss of generality, say $a=p / q$ in lowest terms for some integers $p$ and $q$ - that is, $\operatorname{gcd}(p, q)=1$. Now $q \cdot \frac{p}{q}=p \in H$, and since $q / p \in H$, we have $p \cdot \frac{q}{p} \in H$. There exist integers $x, y$ such that $q x+p y=1$; note that $q x \in H$ and $p y \in H$, so that $1 \in H$. Thus $n \in H$ for all $n \in \mathbb{Z}$. Moreover, if $n \neq 0,1 / n \in H$. Then $m / n \in H$ for all integers $m, n$ with $n \neq 0$; hence $H=\mathbb{Q}$.
\end{proof}



\paragraph{Exercise 2.4.4} Prove that if $H$ is a subgroup of $G$ then $H$ is generated by the set $H-\{1\}$.
\begin{proof}
    If $H=\{1\}$ then $H-\{1\}$ is the empty set which indeed generates the trivial subgroup $H$. So suppose $|H|>1$ and pick a nonidentity element $h \in H$. Since $1=h h^{-1} \in\langle H-\{1\}\rangle$ (Proposition 9), we see that $H \leq\langle H-\{1\}\rangle$. By minimality of $\langle H-\{1\}\rangle$, the reverse inclusion also holds so that $\langle H-\{1\}\rangle=$ $H$.
\end{proof}



\paragraph{Exercise 2.4.16a} A subgroup $M$ of a group $G$ is called a maximal subgroup if $M \neq G$ and the only subgroups of $G$ which contain $M$ are $M$ and $G$. Prove that if $H$ is a proper subgroup of the finite group $G$ then there is a maximal subgroup of $G$ containing $H$.
\begin{proof}
If $H$ is maximal, then we are done. If $H$ is not maximal, then there is a subgroup $K_1$ of $G$ such that $H<K_1<G$. If $K_1$ is maximal, we are done. But if $K_1$ is not maximal, there is a subgroup $K_2$ with $H<K_1<K_2<G$. If $K_2$ is maximal, we are done, and if not, keep repeating the procedure. Since $G$ is finite, this process must eventually come to an end, so that $K_n$ is maximal for some positive integer $n$. Then $K_n$ is a maximal subgroup containing $H$.
\end{proof}



\paragraph{Exercise 2.4.16b} Show that the subgroup of all rotations in a dihedral group is a maximal subgroup.
\begin{proof}
    Fix a positive integer $n>1$ and let $H \leq D_{2 n}$ consist of the rotations of $D_{2 n}$. That is, $H=\langle r\rangle$. Now, this subgroup is proper since it does not contain $s$. If $H$ is not maximal, then by the previous proof we know there is a maximal subset $K$ containing $H$. Then $K$ must contain a reflection $s r^k$ for $k \in\{0,1, \ldots, n-1\}$. Then since $s r^k \in K$ and $r^{n-k} \in K$, it follows by closure that
$$
s=\left(s r^k\right)\left(r^{n-k}\right) \in K .
$$
But $D_{2 n}=\langle r, s\rangle$, so this shows that $K=D_{2 n}$, which is a contradiction. Therefore $H$ must be maximal.
\end{proof}



\paragraph{Exercise 2.4.16c} Show that if $G=\langle x\rangle$ is a cyclic group of order $n \geq 1$ then a subgroup $H$ is maximal if and only $H=\left\langle x^{p}\right\rangle$ for some prime $p$ dividing $n$.
\begin{proof}
    Suppose $H$ is a maximal subgroup of $G$. Then $H$ is cyclic, and we may write $H=\left\langle x^k\right\rangle$ for some integer $k$, with $k>1$. Let $d=(n, k)$. Since $H$ is a proper subgroup, we know by Proposition 6 that $d>1$. Choose a prime factor $p$ of $d$. If $k=p=d$ then $k \mid n$ as required.

If, however, $k$ is not prime, then consider the subgroup $K=\left\langle x^p\right\rangle$. Since $p$ is a proper divisor of $k$, it follows that $H<K$. But $H$ is maximal, so we must have $K=G$. Again by Proposition 6 , we must then have $(p, n)=1$. However, $p$ divides $d$ which divides $n$, so $p \mid n$ and $(p, n)=p>1$, a contradiction. Therefore $k=p$ and the left-to-right implication holds.
Now, for the converse, suppose $H=\left\langle x^p\right\rangle$ for $p$ a prime dividing $n$. If $H$ is not maximal then the first part of this exercise shows that there is a maximal subgroup $K$ containing $H$. Then $K=\left\langle x^q\right\rangle$. So $x^p \in\left\langle x^q\right\rangle$, which implies $q \mid p$. But the only divisors of $p$ are 1 and $p$. If $q=1$ then $K=G$ and $K$ cannot be a proper subgroup, and if $q=p$ then $H=K$ and $H$ cannot be a proper subgroup of $K$. This contradiction shows that $H$ is maximal.
\end{proof}



\paragraph{Exercise 3.1.3a} Let $A$ be an abelian group and let $B$ be a subgroup of $A$. Prove that $A / B$ is abelian.
\begin{proof}
    Lemma: Let $G$ be a group. If $|G|=2$, then $G \cong Z_2$.
Proof: Since $G=\{e a\}$ has an identity element, say $e$, we know that $e e=e, e a=a$, and $a e=a$. If $a^2=a$, we have $a=e$, a contradiction. Thus $a^2=e$. We can easily see that $G \cong Z_2$.

If $A$ is abelian, every subgroup of $A$ is normal; in particular, $B$ is normal, so $A / B$ is a group. Now let $x B, y B \in A / B$. Then
$$
(x B)(y B)=(x y) B=(y x) B=(y B)(x B) .
$$
Hence $A / B$ is abelian.
\end{proof}



\paragraph{Exercise 3.1.22a} Prove that if $H$ and $K$ are normal subgroups of a group $G$ then their intersection $H \cap K$ is also a normal subgroup of $G$.
\begin{proof}
    Suppose $H$ and $K$ are normal subgroups of $G$. We already know that $H \cap K$ is a subgroup of $G$, so we need to show that it is normal. Choose any $g \in G$ and any $x \in H \cap K$. Since $x \in H$ and $H \unlhd G$, we know $g x g^{-1} \in H$. Likewise, since $x \in K$ and $K \unlhd G$, we have $g x g^{-1} \in K$. Therefore $g x g^{-1} \in H \cap K$. This shows that $g(H \cap K) g^{-1} \subseteq H \cap K$, and this is true for all $g \in G$. By Theorem 6 (5) (which we will prove in Exercise 3.1.25), this is enough to show that $H \cap K \unlhd G$.
\end{proof}



\paragraph{Exercise 3.1.22b} Prove that the intersection of an arbitrary nonempty collection of normal subgroups of a group is a normal subgroup (do not assume the collection is countable).
\begin{proof}
Let $\left\{H_i \mid i \in I\right\}$ be an arbitrary collection of normal subgroups of $G$ and consider the intersection
$$
\bigcap_{i \in I} H_i
$$
Take an element $a$ in the intersection and an arbitrary element $g \in G$. Then $g a g^{-1} \in H_i$ because $H_i$ is normal for any $i \in H$
By the definition of the intersection, this shows that $g a g^{-1} \in \bigcap_{i \in I} H_i$ and therefore it is a normal subgroup.
\end{proof}



\paragraph{Exercise 3.2.8} Prove that if $H$ and $K$ are finite subgroups of $G$ whose orders are relatively prime then $H \cap K=1$.
\begin{proof}
    Solution: Let $|H|=p$ and $|K|=q$. We saw in a previous exercise that $H \cap K$ is a subgroup of both $H$ and $K$; by Lagrange's Theorem, then, $|H \cap K|$ divides $p$ and $q$. Since $\operatorname{gcd}(p, q)=1$, then, $|H \cap K|=1$. Thus $H \cap K=1$.
\end{proof}



\paragraph{Exercise 3.2.11} Let $H \leq K \leq G$. Prove that $|G: H|=|G: K| \cdot|K: H|$ (do not assume $G$ is finite).
\begin{proof}
    Proof. Let $G$ be a group and let $I$ be a nonempty set of indices, not necessarily countable. Consider the collection of subgroups $\left\{N_\alpha \mid \alpha \in I\right\}$, where $N_\alpha \unlhd G$ for each $\alpha \in I$. Let
$$
N=\bigcap_{\alpha \in I} N_\alpha .
$$
We know $N$ is a subgroup of $G$. 
For any $g \in G$ and any $n \in N$, we must have $n \in N_\alpha$ for each $\alpha$. And since $N_\alpha \unlhd G$, we have $g n g^{-1} \in N_\alpha$ for each $\alpha$. Therefore $g n g^{-1} \in N$, which shows that $g N g^{-1} \subseteq N$ for each $g \in G$. As before, this is enough to complete the proof.
\end{proof}



\paragraph{Exercise 3.2.16} Use Lagrange's Theorem in the multiplicative group $(\mathbb{Z} / p \mathbb{Z})^{\times}$to prove Fermat's Little Theorem: if $p$ is a prime then $a^{p} \equiv a(\bmod p)$ for all $a \in \mathbb{Z}$.
\begin{proof}
    Solution: If $p$ is prime, then $\varphi(p)=p-1$ (where $\varphi$ denotes the Euler totient). Thus
$$
\mid\left((\mathbb{Z} /(p))^{\times} \mid=p-1 .\right.
$$
So for all $a \in(\mathbb{Z} /(p))^{\times}$, we have $|a|$ divides $p-1$. Hence
$$
a=1 \cdot a=a^{p-1} a=a^p \quad(\bmod p) .
$$
\end{proof}



\paragraph{Exercise 3.2.21a} Prove that $\mathbb{Q}$ has no proper subgroups of finite index.
\begin{proof}
    Solution: We begin with a lemma.
Lemma: If $D$ is a divisible abelian group, then no proper subgroup of $D$ has finite index.
Proof: We saw previously that no finite group is divisible and that every proper quotient $D / A$ of a divisible group is divisible; thus no proper quotient of a divisible group is finite. Equivalently, $[D: A]$ is not finite.
Because $\mathbb{Q}$ and $\mathbb{Q} / \mathbb{Z}$ are divisible, the conclusion follows.
\end{proof}



\paragraph{Exercise 3.3.3} Prove that if $H$ is a normal subgroup of $G$ of prime index $p$ then for all $K \leq G$ either $K \leq H$, or $G=H K$ and $|K: K \cap H|=p$.
\begin{proof}
    Solution: Suppose $K \backslash N \neq \emptyset$; say $k \in K \backslash N$. Now $G / N \cong \mathbb{Z} /(p)$ is cyclic, and moreover is generated by any nonidentity- in particular by $\bar{k}$

Now $K N \leq G$ since $N$ is normal. Let $g \in G$. We have $g N=k^a N$ for some integer a. In particular, $g=k^a n$ for some $n \in N$, hence $g \in K N$. We have $[K: K \cap N]=p$ by the Second Isomorphism Theorem.
\end{proof}



\paragraph{Exercise 3.4.1} Prove that if $G$ is an abelian simple group then $G \cong Z_{p}$ for some prime $p$ (do not assume $G$ is a finite group).
\begin{proof}
    Solution: Let $G$ be an abelian simple group.
Suppose $G$ is infinite. If $x \in G$ is a nonidentity element of finite order, then $\langle x\rangle<G$ is a nontrivial normal subgroup, hence $G$ is not simple. If $x \in G$ is an element of infinite order, then $\left\langle x^2\right\rangle$ is a nontrivial normal subgroup, so $G$ is not simple.

Suppose $G$ is finite; say $|G|=n$. If $n$ is composite, say $n=p m$ for some prime $p$ with $m \neq 1$, then by Cauchy's Theorem $G$ contains an element $x$ of order $p$ and $\langle x\rangle$ is a nontrivial normal subgroup. Hence $G$ is not simple. Thus if $G$ is an abelian simple group, then $|G|=p$ is prime. We saw previously that the only such group up to isomorphism is $\mathbb{Z} /(p)$, so that $G \cong \mathbb{Z} /(p)$. Moreover, these groups are indeed simple.
\end{proof}



\paragraph{Exercise 3.4.4} Use Cauchy's Theorem and induction to show that a finite abelian group has a subgroup of order $n$ for each positive divisor $n$ of its order.
\begin{proof}
    Let $G$ be a finite abelian group. We use induction on $|G|$. Certainly the result holds for the trivial group. And if $|G|=p$ for some prime $p$, then the positive divisors of $|G|$ are 1 and $p$ and the result is again trivial.

Now assume that the statement is true for all groups of order strictly smaller than $|G|$, and let $n$ be a positive divisor of $|G|$ with $n>1$. First, if $n$ is prime then Cauchy's Theorem allows us to find an element $x \in G$ having order $n$. Then $\langle x\rangle$ is the desired subgroup. On the other hand, if $n$ is not prime, then $n$ has a prime divisor $p$, so that $n=k p$ for some integer $k$. Cauchy's Theorem allows us to find an element $x$ having order $p$. Set $N=\langle x\rangle$. By Lagrange's Theorem,
$$
|G / N|=\frac{|G|}{|N|}<|G| .
$$
Now, by the inductive hypothesis, the group $G / N$ must have a subgroup of order $k$. And by the Lattice Isomorphism Theorem, this subgroup has the form $H / N$ for some subgroup $H$ of $G$. Then $|H|=k|N|=k p=n$, so that $H$ has order $n$. This completes the inductive step.
\end{proof}



\paragraph{Exercise 3.4.5a} Prove that subgroups of a solvable group are solvable.
\begin{proof}
    Let $G$ be a solvable group and let $H \leq G$. Since $G$ is solvable, we may find a chain of subgroups
$$
1=G_0 \unlhd G_1 \unlhd G_2 \unlhd \cdots \unlhd G_n=G
$$
so that each quotient $G_{i+1} / G_i$ is abelian. For each $i$, define
$$
H_i=G_i \cap H, \quad 0 \leq i \leq n .
$$
Then $H_i \leq H_{i+1}$ for each $i$. Moreover, if $g \in H_{i+1}$ and $x \in H_i$, then in particular $g \in G_{i+1}$ and $x \in G_i$, so that
$$
g x g^{-1} \in G_i
$$
because $G_i \unlhd G_{i+1}$. But $g$ and $x$ also belong to $H$, so
$$
g x g^{-1} \in H_i,
$$
which shows that $H_i \unlhd H_{i+1}$ for each $i$.
Next, note that
$$
H_i=G_i \cap H=\left(G_i \cap G_{i+1}\right) \cap H=G_i \cap H_{i+1} .
$$
By the Second Isomorphism Theorem, we then have
$$
H_{i+1} / H_i=H_{i+1} /\left(H_{i+1} \cap G_i\right) \cong H_{i+1} G_i / G_i \leq G_{i+1} / G_i .
$$
Since $H_{i+1} / H_i$ is isomorphic to a subgroup of the abelian group $G_{i+1} / G_i$, it follows that $H_{i+1} / H_i$ is also abelian. This completes the proof that $H$ is solvable.
\end{proof}



\paragraph{Exercise 3.4.5b} Prove that quotient groups of a solvable group are solvable.
\begin{proof}
    Next, note that
$$
H_i=G_i \cap H=\left(G_i \cap G_{i+1}\right) \cap H=G_i \cap H_{i+1} .
$$
By the Second Isomorphism Theorem, we then have
$$
H_{i+1} / H_i=H_{i+1} /\left(H_{i+1} \cap G_i\right) \cong H_{i+1} G_i / G_i \leq G_{i+1} / G_i .
$$
Since $H_{i+1} / H_i$ is isomorphic to a subgroup of the abelian group $G_{i+1} / G_i$, it follows that $H_{i+1} / H_i$ is also abelian. This completes the proof that $H$ is solvable.
Next, let $N \unlhd G$. For each $i$, define
$$
N_i=G_i N, \quad 0 \leq i \leq n .
$$
Now let $g \in N_{i+1}$, where $g=g_0 n_0$ with $g_0 \in G_{i+1}$ and $n_0 \in N$. Also let $x \in N_i$, where $x=g_1 n_1$ with $g_1 \in G_i$ and $n_1 \in N$. Then
$$
g x g^{-1}=g_0 n_0 g_1 n_1 n_0^{-1} g_0^{-1} .
$$
Now, since $N$ is normal in $G, N g=g N$, so $n_0 g_1=g_1 n_2$ for some $n_2 \in N$. Then
$$
g x g^{-1}=g_0 g_1\left(n_2 n_1 n_0^{-1}\right) g_0^{-1}=g_0 g_1 n_3 g_0^{-1}
$$
for some $n_3 \in N$. Then $n_3 g_0^{-1}=g_0^{-1} n_4$ for some $n_4 \in N$. And $g_0 g_1 g_0^{-1} \in G_i$ since $G_i \unlhd G_{i+1}$, so
$$
g x g^{-1}=g_0 g_1 g_0^{-1} n_4 \in N_i .
$$
This shows that $N_i \unlhd N_{i+1}$. So by the Lattice Isomorphism Theorem, we have $N_{i+1} / N \unlhd N_i / N$. This shows that
$$
1=N_0 / N \unlhd N_1 / N \unlhd N_2 / N \unlhd \cdots \unlhd N_n / N=G / N .
$$
Moreover, the Third Isomorphism Theorem says that
$$
\left(N_{i+1} / N\right) /\left(N_i / N\right) \cong N_{i+1} / N_i,
$$
so the proof will be complete if we can show that $N_{i+1} / N_i$ is abelian.
Let $x, y \in N_{i+1} / N_i$. Then
$$
x=\left(g_0 n_0\right) N_i \quad \text { and } \quad y=\left(g_1 n_1\right) N_i
$$
for some $g_0, g_1 \in G_{i+1}$ and $n_0, n_1 \in N$. We have
$$
\begin{aligned}
x y x^{-1} y^{-1} & =\left(g_0 n_0\right)\left(g_1 n_1\right)\left(g_0 n_0\right)^{-1}\left(g_1 n_1\right)^{-1} N_i \\
& =g_0 n_0 g_1 n_1 n_0^{-1} g_0^{-1} n_1^{-1} g_1^{-1} N_i .
\end{aligned}
$$
Since $N \unlhd G, g N=N g$ for any $g \in G$, so we can find $n_2 \in N$ such that
$$
x y x^{-1} y^{-1}=g_0 g_1 g_0^{-1} g^{-1} n_2 N_i .
$$
Now $N_i=G_i N=N G_i$ since $N \unlhd G$ (see Proposition 14 and its corollary). Therefore
$$
n_2 N_i=n_2 N G_i=N G_i=G_i N
$$
and we get
$$
x y x^{-1} y^{-1}=g_0 g_1 g_0^{-1} g^{-1} G_i N=G_i N .
$$
So $x y x^{-1} y^{-1}=1 N_i$ or $x y=y x$. This completes the proof that $G / N$ is solvable.
\end{proof}



\paragraph{Exercise 3.4.11} Prove that if $H$ is a nontrivial normal subgroup of the solvable group $G$ then there is a nontrivial subgroup $A$ of $H$ with $A \unlhd G$ and $A$ abelian.
\begin{proof}
    Suppose $H$ is a nontrivial normal subgroup of the solvable group $G$.
First, notice that $H$, being a subgroup of a solvable group, is itself solvable. By exercise $8, H$ has a chain of subgroups
$$
1 \leq H_1 \leq \ldots \leq H
$$
such that each $H_i$ is a normal subgroup of $H$ itself and $H_{i+1} / H_i$ is abelian. But then the first group in the series
$$
H_1 / 1 \cong H
$$
is an abelian subgroup of $H$.
\end{proof}



\paragraph{Exercise 4.2.8} Prove that if $H$ has finite index $n$ then there is a normal subgroup $K$ of $G$ with $K \leq H$ and $|G: K| \leq n!$.
\begin{proof}
    Solution: $G$ acts on the cosets $G / H$ by left multiplication. Let $\lambda: G \rightarrow S_{G / H}$ be the permutation representation induced by this action, and let $K$ be the kernel of the representation.
Now $K$ is normal in $G$, and $K \leq \operatorname{stab}_G(H)=H$. By the First Isomorphism Theorem, we have an injective group homomorphism $\bar{\lambda}: G / K \rightarrow S_{G / H}$. Since $\left|S_{G / H}\right|=n !$, we have $[G: K] \leq n !$.
\end{proof}



\paragraph{Exercise 4.2.9a} Prove that if $p$ is a prime and $G$ is a group of order $p^{\alpha}$ for some $\alpha \in \mathbb{Z}^{+}$, then every subgroup of index $p$ is normal in $G$.
\begin{proof}
    Solution: Let $G$ be a group of order $p^k$ and $H \leq G$ a subgroup with $[G: H]=p$. Now $G$ acts on the conjugates $g H g^{-1}$ by conjugation, since
$$
g_1 g_2 \cdot H=\left(g_1 g_2\right) H\left(g_1 g_2\right)^{-1}=g_1\left(g_2 H g_2^{-1}\right) g_1^{-1}=g_1 \cdot\left(g_2 \cdot H\right)
$$
and $1 \cdot H=1 H 1=H$. Moreover, under this action we have $H \leq \operatorname{stab}(H)$. By Exercise 3.2.11, we have
$$
[G: \operatorname{stab}(H)][\operatorname{stab}(H): H]=[G: H]=p,
$$
a prime.
If $[G: \operatorname{stab}(H)]=p$, then $[\operatorname{stab}(H): H]=1$ and we have $H=\operatorname{stab}(H)$; moreover, $H$ has exactly $p$ conjugates in $G$. Let $\varphi: G \rightarrow S_p$ be the permutation representation induced by the action of $G$ on the conjugates of $H$, and let $K$ be the kernel of this representation. Now $K \leq \operatorname{stab}(H)=H$. By the first isomorphism theorem, the induced map $\bar{\varphi}: G / K \rightarrow S_p$ is injective, so that $|G / K|$ divides $p$ !. Note, however, that $|G / K|$ is a power of $p$ and that the only powers of $p$ that divide $p$ ! are 1 and $p$. So $[G: K]$ is 1 or $p$. If $[G: K]=1$, then $G=K$ so that $g H g^{-1}=H$ for all $g \in G$; then $\operatorname{stab}(H)=G$ and we have $[G: \operatorname{stab}(H)]=1$, a contradiction. Now suppose $[G: K]=p$. Again by Exercise $3.2$.11 we have $[G: K]=[G: H][H: K]$, so that $[H: K]=1$, hence $H=K$. Again, this implies that $H$ is normal so that $g H g^{-1}=H$ for all $g \in G$, and we have $[G: \operatorname{stab}(H)]=1$, a contradiction. Thus $[G: \operatorname{stab}(H)] \neq p$
If $[G: \operatorname{stab}(H)]=1$, then $G=\operatorname{stab}(H)$. That is, $g H g^{-1}=H$ for all $g \in G$; thus $H \leq G$ is normal.
\end{proof}



\paragraph{Exercise 4.2.14} Let $G$ be a finite group of composite order $n$ with the property that $G$ has a subgroup of order $k$ for each positive integer $k$ dividing $n$. Prove that $G$ is not simple.
\begin{proof}
    Solution: Let $p$ be the smallest prime dividing $n$, and write $n=p m$. Now $G$ has a subgroup $H$ of order $m$, and $H$ has index $p$. Then $H$ is normal in $G$.
\end{proof}



\paragraph{Exercise 4.3.26} Let $G$ be a transitive permutation group on the finite set $A$ with $|A|>1$. Show that there is some $\sigma \in G$ such that $\sigma(a) \neq a$ for all $a \in A$.
\begin{proof}
    Let $G$ be a transitive permutation group on the finite set $A,|A|>1$. We want to find an element $\sigma$ which doesn't stabilize anything, that is, we want a $\sigma$ such that
$$
\sigma \notin G_a
$$
for all $a \in A$.
Since the group is transitive, there is always a $g \in G$ such that $b=g \cdot a$. Let us see in what relationship the stabilizers of $a$ and $b$ are. We find
$$
\begin{aligned}
G_b & =\{h \in G \mid h \cdot b=b\} \\
& =\{h \in G \mid h g \cdot a=g \cdot a\} \\
& =\left\{h \in G \mid g^{-1} h g \cdot a=a\right\}
\end{aligned}
$$
Putting $h^{\prime}=g^{-1} h g$, we have $h=g h^{\prime} g^{-1}$ and
$$
\begin{aligned}
G_b & =g\left\{h^{\prime} \in H \mid h^{\prime} \cdot a=a\right\} g^{-1} \\
& =g G_a g^{-1}
\end{aligned}
$$
By the above, the stabilizer subgroup of any element is conjugate to some other stabilizer subgroup. Now, the stabilizer cannot be all of $G$ (else $\{a\}$ would be a orbit). Thus it is a proper subgroup of $G$. By the previous exercise, we have
$$
\bigcup_{a \in A} G_a=\bigcup_{g \in G} g G_a g^{-1} \subset G
$$
(the union of conjugates of a proper subgroup can never be all of $G$ ). This shows there is an element $\sigma$ which is not in any stabilizer of any element of $A$. Then $\sigma(a) \neq a$ for all $a \in A$, as we wanted to show.
\end{proof}



\paragraph{Exercise 4.4.2} Prove that if $G$ is an abelian group of order $p q$, where $p$ and $q$ are distinct primes, then $G$ is cyclic.
\begin{proof}
    Let $G$ be an abelian group of order $p q$. We need to prove that if $p$ and $q$ are distinct primes than $G$ is cyclic. By Cauchy's theorem there are $a, b \in G$ with $a$ of order $p$ and $b$ of order $q$. Since $(|a|,|b|)=1$ and $a b=b a$ then $|a b|=|a| \cdot|b|=p q$. Therefore $a b$ is an element of order $p q$, the order of $G$, which means $G$ is cyclic.
\end{proof}



\paragraph{Exercise 4.4.6a} Prove that characteristic subgroups are normal.
\begin{proof}
    Let $H$ be a characteristic subgroup of $G$. By definition $\alpha(H) \subset H$ for every $\alpha \in \operatorname{Aut}(G)$. So, $H$ is in particular invariant under the inner automorphism. Let $\phi_g$ denote the conjugation automorphism by $g$. Then $\phi_g(H) \subset H \Longrightarrow$ $g H g^{-1} \subset H$. So, $H$ is normal. 
\end{proof}



\paragraph{Exercise 4.4.6b} Prove that there exists a normal subgroup that is not characteristic.
\begin{proof}
    We have to produce a group $G$ and a subgroup $H$ such that $H$ is normal in $G$, but not characteristic. Consider the Klein's four group $G=\{ e, a, b, a b\}$. This is an abelian group with each element having order 2. Consider $H=\{ e, a\}$. $H$ is normal in $G$. Define $\sigma: G \rightarrow G$ as $\sigma(a)=b, \sigma(b)=a, \sigma(a b)=a b$. Clearly $\sigma$ does not fix $H$. So, $H$ is not characteristic.
\end{proof}



\paragraph{Exercise 4.4.7} If $H$ is the unique subgroup of a given order in a group $G$ prove $H$ is characteristic in $G$.
\begin{proof}
    Let $G$ be group and $H$ be the unique subgroup of order $n$. Now, let $\sigma \in \operatorname{Aut}(G)$. Now Clearly $|\sigma(G)|=n$, because $\sigma$ is a one-one onto map. But then as $H$ is the only subgroup of order $n$, and because of the fact that a automorphism maps subgroups to subgroups, we have $\sigma(H)=$ $H$ for every $\sigma \in \operatorname{Aut}(G)$. Hence, $H$ is a characteristic subgroup of $G$.
\end{proof}



\paragraph{Exercise 4.4.8a} Let $G$ be a group with subgroups $H$ and $K$ with $H \leq K$. Prove that if $H$ is characteristic in $K$ and $K$ is normal in $G$ then $H$ is normal in $G$.
\begin{proof}
We prove that $H$ is invariant under every inner automorphism of $G$. Consider a inner automorphism $\phi_g$ of $G$. Now, $\left.\phi_g\right|_K$ is a automorphism of $K$ because $K$ is normal in $G$. But $H$ is a characteristic subgroup of $K$, so $\left.\phi_g\right|_K(H) \subset H$, so in general $\phi_g(H) \subset H$. Hence $H$ is characteristic in $G$.
\end{proof}



\paragraph{Exercise 4.5.1a} Prove that if $P \in \operatorname{Syl}_{p}(G)$ and $H$ is a subgroup of $G$ containing $P$ then $P \in \operatorname{Syl}_{p}(H)$.
\begin{proof}
If $P \leq H \leq G$ is a Sylow $p$-subgroup of $G$, then $p$ does not divide $[G: P]$. Now $[G: P]=[G: H][H: P]$, so that $p$ does not divide $[H: P]$; hence $P$ is a Sylow $p$-subgroup of $H$.
\end{proof}



\paragraph{Exercise 4.5.13} Prove that a group of order 56 has a normal Sylow $p$-subgroup for some prime $p$ dividing its order.
\begin{proof}    
Since $|G|=56=2^{3}.7$, $G$ has $2-$Sylow subgroup of order $8$, as well as $7-$Sylow subgroup of order $7$. Now, we count the number of such subgroups. Let $n_{7}$ be the number of  $7-$Sylow subgroup and $n_{2}$ be the number of  $2-$Sylow subgroup. Now $n_{7}=1+7k$ where $1+7k|8$. The choices for $k$ are $0$ or $1$. If $k=0$, there is only one $7-$Sylow subgroup and hence normal. So, assume now, that there are $8$ $7-$Sylow subgroup(for $k=1$). Now we look at $2-$ Sylow subgroups. $n_{2}=1+2k| 7$. So choice for $k$ are $0$ and $3$. If $k=0$, there is only one $2-$Sylow subgroup and hence normal. So, assume now, that there are $7$ $2-$Sylow subgroup (for $k=3$). Now we claim that simultaneously, there cannot be $8$ $7-$Sylow subgroup and $7$ $2-$Sylow subgroup. So, either $7-$Sylow subgroup is normal being unique, or  the $2-$Sylow subgroup is normal. Now, to prove the claim, we observe that there are 48 elements of order $7$. Let $H_{1}$ and $H_{2}$ be two distinct  $2-$Sylow subgroup. Now $|H_{1}|=8$. So we already get $48+8=56$ distinct elements in the group. Now $H_{2}$ being distinct from $H_{1}$, has at least one element which is not in $H_{1}$. This adds one more element in the group, at the least. Now already we have number of elements in the group exceeding the number of element in $G$. This gives a contradiction and proves the claim.
\end{proof}



\paragraph{Exercise 4.5.14} Prove that a group of order 312 has a normal Sylow $p$-subgroup for some prime $p$ dividing its order.
\begin{proof}
Let $n_{13}$ be the number of Sylow 13 -subgroup of $G$. Then by Sylow's Theorem, $n_{13} \equiv 1(\bmod 13)$ and $n_{13}$ divides $2^3 \cdot 3=24$. This implies $n_{13}=1$, so that there is only one Sylow 13 -subgroup, which is consequently normal. The last assertion follows from the fact conjugation preserves the order of a subgroup. So if there is only one subgroup $H$ of order 13 , then for any $g \in G$, we have $\left|g H g^{-1}\right|=|H|=13$, so $g H g^{-1}=H$, i.e. $H$ is normal.
\end{proof}



\paragraph{Exercise 4.5.15} Prove that a group of order 351 has a normal Sylow $p$-subgroup for some prime $p$ dividing its order.
\begin{proof}
    Since $|G|=351=3^{2}.13$, $G$ has $3-$Sylow subgroup of order $9$, as well as $13-$Sylow subgroup of order $13$. Now, we count the number of such subgroups. Let $n_{13}$ be the number of $13-$Sylow subgroup and $n_{3}$ be the number of  $3-$Sylow subgroup. Now $n_{13}=1+13k$ where $1+13k|9$. The choices for $k$ is $0$. Hence, there is a unique $13-$Sylow subgroup and hence is normal.
\end{proof}



\paragraph{Exercise 4.5.16} Let $|G|=p q r$, where $p, q$ and $r$ are primes with $p<q<r$. Prove that $G$ has a normal Sylow subgroup for either $p, q$ or $r$.
\begin{proof}
    Let $|G|=p q r$. We also assume $p<q<r$. We prove that $G$ has a normal Sylow subgroup of $p$, $q$ or $r$. Now, Let $n_p, n_q, n_r$ be the number of Sylow-p subgroup, Sylow-q subgroup, Sylow-r subgroup resp. So, we have $n_r=1+r k$ such that $1+r k \mid p q$. So, in this case as $r$ is greatest $n_r$ can be 1 or $p q$. We assume $n_r=p q$. Now we have $n_q=1+q k$ such that $1+q k \mid p r$. Now,as $p<q<r, n_q$ can be 1 or $r$, or $p r$. Assume that $n_q=r$. Now we turn to $n_p$. Again my similar method we can conclude $n_p$ can be $1, q, r$, or $q r$. We assume that $n_p$ is $q$. Now we count the number of elements of order $p, q, r$. Since $n_r=p q$, the number of elements of order $r$ is $p q(r-1)$. Since $n_q=r$, the number of elements of order $q$ is $(q-1) r$. And as $n_p=q$, the number of elements of order $p$ is $(p-1) q$. So, in total we get $p q(r-1)+(q-1) r+(p-$ 1) $q=p q r+q r-r-q=p q r+r(q-1)-r$. But observe that as $q>1, r(q-1)-r>$ 0 . So the number of elements exceeds $p q r$. So, it proves that at least $n_p$ or $n_q$ or $n_r$ is 1, which ultimately proves the result, because a unique Sylow-p subgroup is always normal.
\end{proof}



\paragraph{Exercise 4.5.17} Prove that if $|G|=105$ then $G$ has a normal Sylow 5 -subgroup and a normal Sylow 7-subgroup.
\begin{proof}    
Since $|G|=105=3.5.7$, $G$ has $3-$Sylow subgroup of order $3$, as well as $5-$Sylow subgroup of order $5$ and, $7-$Sylow subgroup of order 7. Now, we count the number of such subgroups. Let $n_{3}$ be the number of $3-$Sylow subgroup, $n_{5}$ be the number of  $5-$Sylow subgroup, and $n_{7}$ be the number of $7-$Sylow subgroup. Now $n_{7}=1+7k$ where $1+7k|15$. The choices for $k$ are $0$ or $1$. If $k=0$, there is only one $7-$Sylow subgroup and hence normal. So, assume now, that there are $15$ $7-$Sylow subgroup(for $k=1$). Now we look at $5-$ Sylow subgroups. $n_{5}=1+5k| 21$. So choice for $k$ are $0$ and $4$. If $k=0$, there is only one $5-$Sylow subgroup and hence normal. So, assume now, that there are $24$ $5-$Sylow subgroup (for $k=4$). Now we claim that simultaneously, there cannot be $15$ $7-$Sylow subgroup and $24$ $5-$Sylow subgroup. So, either $7-$Sylow subgroup is normal being unique, or  the $5-$Sylow subgroup is normal. Now, to prove the claim, we observe that there are 90 elements of order $7$. Also, see that there are $24\times 4=96$ number of elements of order 5. So we get $90+94=184$ number of elements which exceeds the order of the group. This gives a contradiction and proves the claim. So, now we have proved that there is either a normal $5-$Sylow subgroup or a normal $7-$Sylow subgroup.
    Now we prove that indeed both $5-$ Sylow subgroup and 7 -Sylow subgroup are normal. Assume that 7 -Sylow subgroup is normal. So, there is a unique 7 -Sylow subgroup, say $H$. Now assume that there are 245 -Sylow subgroups. So, we get again $24 \times 4=96$ elements of order 5 . From $H$ we get 7 elements which gives us total of $96+7=103$ elements. Now consider the number of 3 -Sylow subgroups. $n_3=1+3 k \mid 35$. Then the possibilities for $k$ are 0 and 2 . But we can rule out $k=2$ because having 73 -Sylow subgroup, will mean we have 14 elements of order 3 . So we get $103+14=117$ elements in total which exceeds the order of the group. So we have now that there is a unique 3 -Sylow subgroup and hence normal. Call that subgroup $K$. Now take any one 5 -Sylow subgroup, call it $L$. Now observe $L K$ is a subgroup of $G$ with order 15 . We know that a group of order 15 is cyclic by an example in Page-143 of the book. So, there is an element of order 15. Actually we have $\phi(15)=8$ number of elements of order 15. But then again we already had 103 elements and then we actually get at least $103+8=111$ elements which exceeds the order of the group. So, there can't be 24 5-Sylow subgroups, and hence there is a unique 5-Sylow subgroup, and hence normal.
\end{proof}



\paragraph{Exercise 4.5.18} Prove that a group of order 200 has a normal Sylow 5-subgroup.
\begin{proof}
    Let $G$ be a group of order $200=5^2 \cdot 8$. Note that 5 is a prime not dividing 8 . Let $P \in$ $S y l_5(G)$. [We know $P$ exists since $S y l_5(G) \neq \emptyset$ by Sylow's Theorem]

The number of Sylow 5-subgroups of $G$ is of the form $1+k \cdot 5$, i.e., $n_5 \equiv 1(\bmod 5)$ and $n_5$ divides 8 . The only such number that divides 8 and equals $1 (\bmod 5)$ is 1 so $n_5=1$. Hence $P$ is the unique Sylow 5-subgroup.
Since $P$ is the unique Sylow 5-subgroup, this implies that $P$ is normal in $G$.
\end{proof}


\paragraph{Exercise 4.5.19} Prove that if $|G|=6545$ then $G$ is not simple.
\begin{proof}    
Since $|G|=132=2^{2}.3.11$, $G$ has $2-$Sylow subgroup of order $4$, as well as $11-$Sylow subgroup of order $11$, and $3-$Sylow subgroup of order $3$. Now, we count the number of such subgroups. Let $n_{11}$ be the number of  $11-$Sylow subgroup and $n_{3}$ be the number of  $3-$Sylow subgroup. Now $n_{11}=1+11k$ where $1+11k|12$. The choices for $k$ are $0$ or $1$. If $k=0$, there is only one $11-$Sylow subgroup and hence normal. So, assume now, that there are $12$ $11-$Sylow subgroup(for $k=1$). Now we look at $3-$ Sylow subgroups. $n_{3}=1+3k| 44$. So choice for $k$ are $0$, $1$, and $7$. If $k=0$, there is only one $3-$Sylow subgroup and hence normal. So, assume now, that there are $4$ $2-$Sylow subgroup (for $k=3$). Now we claim that simultaneously, there cannot be $12$ $11-$Sylow subgroup and $4$ $3-$Sylow subgroups provided there is more than one $2-$Sylow subgroups. So, either $2-$Sylow subgroup is normal or if not, then, either $11-$Sylow subgroup is normal being unique, or  the $3-$Sylow subgroup is normal(We don't consider the possibility of $22$ $3-$Sylow subgroup because of obvious reason). Now, to prove the claim, we observe that there are $120$ elements of order $11$. Also there are $8$ elements of order $3$. So we already get $120+8+1=129$ distinct elements in the group. Let us count the number of $2-$Sylow subgroups in $G$. $n_{2}=1+2k|33$. The possibilities for $k$ are $0$, $1$, $5$, $16$. Now, assume there is more than one $2-$Sylow subgroups. Let $H_{1}$ and $H_{2}$ be two distinct  $2-$Sylow subgroup. Now $|H_{1}|=4$. So we already get $129+3=132$ distinct elements in the group. Now $H_{2}$ being distinct from $H_{1}$, has at least one element which is not in $H_{1}$. This adds one more element in the group, at the least. Now already we have number of elements in the group exceeding the number of element in $G$. This gives a contradiction and proves the claim.
Hence $G$ is not simple.
\end{proof}



\paragraph{Exercise 4.5.20} Prove that if $|G|=1365$ then $G$ is not simple.
\begin{proof}    
Since $|G|=1365=3.5.7.13$, $G$ has $13-$Sylow subgroup of order $13$. Now, we count the number of such subgroups. Let $n_{13}$ be the number of $13-$Sylow subgroup. Now $n_{13}=1+13k$ where $1+13k|3.5.7$. The choices for $k$ is $0$. Hence, there is a unique $13-$Sylow subgroup and hence is normal. so $G$ is not simple.
\end{proof}



\paragraph{Exercise 4.5.21} Prove that if $|G|=2907$ then $G$ is not simple.
\begin{proof}    
Since $|G|=2907=3^{2}.17.19$, $G$ has $19-$Sylow subgroup of order $19$. Now, we count the number of such subgroups. Let $n_{19}$ be the number of $19-$Sylow subgroup. Now $n_{19}=1+19k$ where $1+19k|3^{2}.17$. The choices for $k$ is $0$. Hence, there is a unique $19-$Sylow subgroup and hence is normal. so $G$ is not simple.
\end{proof}



\paragraph{Exercise 4.5.22} Prove that if $|G|=132$ then $G$ is not simple.
\begin{proof}    
Since $|G|=132=2^{2}.3.11$, $G$ has $2-$Sylow subgroup of order $4$, as well as $11-$Sylow subgroup of order $11$, and $3-$Sylow subgroup of order $3$. Now, we count the number of such subgroups. Let $n_{11}$ be the number of  $11-$Sylow subgroup and $n_{3}$ be the number of  $3-$Sylow subgroup. Now $n_{11}=1+11k$ where $1+11k|12$. The choices for $k$ are $0$ or $1$. If $k=0$, there is only one $11-$Sylow subgroup and hence normal. So, assume now, that there are $12$ $11-$Sylow subgroup(for $k=1$). Now we look at $3-$ Sylow subgroups. $n_{3}=1+3k| 44$. So choice for $k$ are $0$, $1$, and $7$. If $k=0$, there is only one $3-$Sylow subgroup and hence normal. So, assume now, that there are $4$ $2-$Sylow subgroup (for $k=3$). Now we claim that simultaneously, there cannot be $12$ $11-$Sylow subgroup and $4$ $3-$Sylow subgroups provided there is more than one $2-$Sylow subgroups. So, either $2-$Sylow subgroup is normal or if not, then, either $11-$Sylow subgroup is normal being unique, or  the $3-$Sylow subgroup is normal(We don't consider the possibility of $22$ $3-$Sylow subgroup because of obvious reason). Now, to prove the claim, we observe that there are $120$ elements of order $11$. Also there are $8$ elements of order $3$. So we already get $120+8+1=129$ distinct elements in the group. Let us count the number of $2-$Sylow subgroups in $G$. $n_{2}=1+2k|33$. The possibilities for $k$ are $0$, $1$, $5$, $16$. Now, assume there is more than one $2-$Sylow subgroups. Let $H_{1}$ and $H_{2}$ be two distinct  $2-$Sylow subgroup. Now $|H_{1}|=4$. So we already get $129+3=132$ distinct elements in the group. Now $H_{2}$ being distinct from $H_{1}$, has at least one element which is not in $H_{1}$. This adds one more element in the group, at the least. Now already we have number of elements in the group exceeding the number of element in $G$. This gives a contradiction and proves the claim.
Hence $G$ is not simple.
\end{proof}



\paragraph{Exercise 4.5.23} Prove that if $|G|=462$ then $G$ is not simple.
\begin{proof}
    Let $G$ be a group of order $462=11 \cdot 42$. Note that 11 is a prime not dividing 42 . Let $P \in$ $S y l_{11}(G)$. [We know $P$ exists since $S y l_{11}(G) \neq \emptyset$]. Note that $|P|=11^1=11$ by definition. 

The number of Sylow 11-subgroups of $G$ is of the form $1+k \cdot 11$, i.e., $n_{11} \equiv 1$ (mod 11) and $n_{11}$ divides 42 . The only such number that divides 42 and equals 1 (mod 11) is 1 so $n_{11}=1$. Hence $P$ is the unique Sylow 11-subgroup.

Since $P$ is the unique Sylow Il-subgroup, this implies that $P$ is normal in $G$.
\end{proof}



\paragraph{Exercise 4.5.28} Let $G$ be a group of order 105. Prove that if a Sylow 3-subgroup of $G$ is normal then $G$ is abelian.
\begin{proof}
    Given that $G$ is a group of order $1575=3^2 .5^2 .7$. Now, Let $n_p$ be the number of Sylow-p subgroups. It is given that Sylow-3 subgroup is normal and hence is unique, so $n_3=1$. First we prove that both Sylow-5 subgroup and Sylow 7-subgroup are normal. Let $P$ be the Sylow3 subgroup. Now, Consider $G / P$, which has order $5^2 .7$. Now, the number of Sylow $-5$ subgroup of $G / P$ is given by $1+5 k$, where $1+5 k \mid 7$. Clearly $k=0$ is the only choice and hence there is a unique Sylow-5 subgroup of $G / P$, and hence normal. In the same way Sylow-7 subgroup of $G / P$ is also unique and hence normal. Consider now the canonical map $\pi: G \rightarrow G / P$. The inverse image of Sylow-5 subgroup of $G / P$ under $\pi$, call it $H$, is a normal subgroup of $G$, and $|H|=3^2 .5^2$. Similarly, the inverse image of Sylow-7 subgroup of $G / P$ under $\pi$ call it $K$ is also normal in $G$ and $|K|=3^2 .7$. Now, consider $H$. Observe first that the number of Sylow-5 subgroup in $H$ is $1+5 k$ such that $1+5 k \mid 9$. Again $k=0$ and hence $H$ has a unique Sylow-5 subgroup, call it $P_1$. But, it is easy to see that $P_1$ is also a Sylow-5 subgroup of $G$, because $\left|P_1\right|=25$. But now any other Sylow 5 subgroup of $G$ is of the form $g P_1 g^{-1}$ for some $g \in G$. But observe that since $P_1 \subset H$ and $H$ is normal in $G$, so $g P_1 g^1 \subset H$, and $g P_1 g^{-1}$ is also Sylow-5 subgroup of $H$. But, then as Sylow-5 subgroup of $H$ is unique we have $g P_1 g^{-1}=P_1$. This shows that Sylow-5 subgroup of $G$ is unique and hence normal in $G$.

Similarly, one can argue the same for $K$ and deduce that Sylow-7 subgroup of $G$ is unique and hence normal. So, the first part of the problem is done.
\end{proof}



\paragraph{Exercise 4.5.33} Let $P$ be a normal Sylow $p$-subgroup of $G$ and let $H$ be any subgroup of $G$. Prove that $P \cap H$ is the unique Sylow $p$-subgroup of $H$.
\begin{proof}
    Let $G$ be a group and $P$ is a normal $p$-Sylow subgroup of $G .|G|=p^a . m$ where $p \nmid m$. Then $|P|=p^a$. Let $H$ be a subgroup of $G$. Now if $|H|=k$ such that $p \nmid k$. Then $P \cap H=\{e\}$. There is nothing to prove in this case. Let $|H|=p^b . n$, where $b \leq a$, and $p \nmid n$. Now consider $P H$ which is a subgroup of $G$, as $P$ is normal. Now $|P H|=\frac{|P||H|}{|P \cap H|}=\frac{p^{a+b} \cdot n}{|P \cap H|}$. Now since $P H \leq G$, so $|P H|=p^a$.l, as $P \leq P H$. This forces $|P \cap H|=p^b$. So by order consideration we have $P \cap H$ is a sylow $-p$ subgroup of $H$. Now we know $P$ is unique $p$ - Sylow subgroup. Suppose $H$ has a sylow-p subgroup distinct from $P \cap H$, call it $H_1$. Now $H_1$ is a p-subgroup of $G$. So, $H_1$ is contained in some Sylow-p subgroup of $G$, call it $P_1$. Clearly $P_1$ is distinct from $P$, which is a contradiction. So $P \cap H$ is the only $p$-Sylow subgroup of $H$, and hence normal in $H$
\end{proof}



\paragraph{Exercise 5.4.2} Prove that a subgroup $H$ of $G$ is normal if and only if $[G, H] \leq H$.
\begin{proof}
    $H \unlhd G$ is equivalent to $g^{-1} h g \in H, \forall g \in G, \forall h \in H$. We claim that holds if and only if $h^{-1} g^{-1} h g \in H, \forall g \in G, \forall h \in H$, i.e., $\left\{h^{-1} g^{-1} h g: h \in H, g \in G\right\} \subseteq H$. That holds by the following argument:
If $g^{-1} h g \in H, \forall g \in G, \forall h \in H$, note that $h^{-1} \in H$, so multiplying them, we also obtain an element of $H$.
On the other hand, if $h^{-1} g^{-1} h g \in H, \forall g \in G, \forall h \in H$, then
$$
h h^{-1} g^{-1} h g=g^{-1} h g \in H, \forall g \in G, \forall h \in H .
$$
Since $\left\{h^{-1} g^{-1} h g: h \in H, g \in G\right\} \subseteq H \Leftrightarrow\left\langle\left\{h^{-1} g^{-1} h g: h \in H, g \in G\right\}\right\rangle \leq H$, we've solved the exercise by definition of $[H, G]$.
\end{proof}



\paragraph{Exercise 7.1.2} Prove that if $u$ is a unit in $R$ then so is $-u$.
\begin{proof}
    Solution: Since $u$ is a unit, we have $u v=v u=1$ for some $v \in R$. Thus, we have
$$
(-v)(-u)=v u=1
$$
and
$$
(-u)(-v)=u v=1 .
$$
Thus $-u$ is a unit.
\end{proof}



\paragraph{Exercise 7.1.11} Prove that if $R$ is an integral domain and $x^{2}=1$ for some $x \in R$ then $x=\pm 1$.
\begin{proof}
    Solution: If $x^2=1$, then $x^2-1=0$. Evidently, then,
$$
(x-1)(x+1)=0 .
$$
Since $R$ is an integral domain, we must have $x-1=0$ or $x+1=0$; thus $x=1$ or $x=-1$.
\end{proof}



\paragraph{Exercise 7.1.12} Prove that any subring of a field which contains the identity is an integral domain.
\begin{proof}
    Solution: Let $R \subseteq F$ be a subring of a field. (We need not yet assume that $1 \in R$ ). Suppose $x, y \in R$ with $x y=0$. Since $x, y \in F$ and the zero element in $R$ is the same as that in $F$, either $x=0$ or $y=0$. Thus $R$ has no zero divisors. If $R$ also contains 1 , then $R$ is an integral domain.
\end{proof}



\paragraph{Exercise 7.1.15} A ring $R$ is called a Boolean ring if $a^{2}=a$ for all $a \in R$. Prove that every Boolean ring is commutative.
\begin{proof}
    Solution: Note first that for all $a \in R$,
$$
-a=(-a)^2=(-1)^2 a^2=a^2=a .
$$
Now if $a, b \in R$, we have
$$
a+b=(a+b)^2=a^2+a b+b a+b^2=a+a b+b a+b .
$$
Thus $a b+b a=0$, and we have $a b=-b a$. But then $a b=b a$. Thus $R$ is commutative.
\end{proof}



\paragraph{Exercise 7.2.2} Let $p(x)=a_{n} x^{n}+a_{n-1} x^{n-1}+\cdots+a_{1} x+a_{0}$ be an element of the polynomial ring $R[x]$. Prove that $p(x)$ is a zero divisor in $R[x]$ if and only if there is a nonzero $b \in R$ such that $b p(x)=0$.
\begin{proof}
    Solution: If $b p(x)=0$ for some nonzero $b \in R$, then it is clear that $p(x)$ is a zero divisor.
Now suppose $p(x)$ is a zero divisor; that is, for some $q(x)=\sum_{i=0}^m b_i x^i$, we have $p(x) q(x)=0$. We may choose $q(x)$ to have minimal degree among the nonzero polynomials with this property.
We will now show by induction that $a_i q(x)=0$ for all $0 \leq i \leq n$.
For the base case, note that
$$
p(x) q(x)=\sum_{k=0}^{n+m}\left(\sum_{i+j=k} a_i b_j\right) x^k=0 .
$$
The coefficient of $x^{n+m}$ in this product is $a_n b_m$ on one hand, and 0 on the other. Thus $a_n b_m=0$. Now $a_n q(x) p(x)=0$, and the coefficient of $x^m$ in $q$ is $a_n b_m=0$. Thus the degree of $a_n q(x)$ is strictly less than that of $q(x)$; since $q(x)$ has minimal degree among the nonzero polynomials which multiply $p(x)$ to 0 , in fact $a_n q(x)=0$. More specifically, $a_n b_i=0$ for all $0 \leq i \leq m$.
For the inductive step, suppose that for some $0 \leq t<n$, we have $a_r q(x)=0$ for all $t<r \leq n$. Now
$$
p(x) q(x)=\sum_{k=0}^{n+m}\left(\sum_{i+j=k} a_i b_j\right) x^k=0 .
$$
On one hand, the coefficient of $x^{m+t}$ is $\sum_{i+j=m+t} a_i b_j$, and on the other hand, it is 0 . Thus
$$
\sum_{i+j=m+t} a_i b_j=0 .
$$
By the induction hypothesis, if $i \geq t$, then $a_i b_j=0$. Thus all terms such that $i \geq t$ are zero. If $i<t$, then we must have $j>m$, a contradiction. Thus we have $a_t b_m=0$. As in the base case,
$$
a_t q(x) p(x)=0
$$
and $a_t q(x)$ has degree strictly less than that of $q(x)$, so that by minimality, $a_t q(x)=0$.
By induction, $a_i q(x)=0$ for all $0 \leq i \leq n$. In particular, $a_i b_m=0$. Thus $b_m p(x)=0$.
\end{proof}


\paragraph{Exercise 7.2.12} Let $G=\left\{g_{1}, \ldots, g_{n}\right\}$ be a finite group. Prove that the element $N=g_{1}+g_{2}+\ldots+g_{n}$ is in the center of the group ring $R G$.
\begin{proof}
    Let $M=\sum_{i=1}^n r_i g_i$ be an element of $R[G]$. Note that for each $g_i \in G$, the action of $g_i$ on $G$ by conjugation permutes the subscripts. Then we have the following.
$$
\begin{aligned}
N M &=\left(\sum_{i=1}^n g_i\right)\left(\sum_{j=1}^n r_j g_j\right) \\
&=\sum_{j=1}^n \sum_{i=1}^n r_j g_i g_j \\
&=\sum_{j=1}^n \sum_{i=1}^n r_j g_j g_j^{-1} g_i g_j \\
&=\sum_{j=1}^n r_j g_j\left(\sum_{i=1}^n g_j^{-1} g_i g_j\right) \\
&=\sum_{j=1}^n r_j g_j\left(\sum_{i=1}^n g_i\right) \\
&=\left(\sum_{j=1}^n r_j g_j\right)\left(\sum_{i=1}^n g_i\right) \\
&=M N .
\end{aligned}
$$
Thus $N \in Z(R[G])$.
\end{proof}



\paragraph{Exercise 7.3.16} Let $\varphi: R \rightarrow S$ be a surjective homomorphism of rings. Prove that the image of the center of $R$ is contained in the center of $S$.
\begin{proof}
    Suppose $r \in \varphi[Z(R)]$. Then $r=\varphi(z)$ for some $z \in Z(R)$. Now let $x \in S$. Since $\varphi$ is surjective, we have $x=\varphi y$ for some $y \in R$. Now
$$
x r=\varphi(y) \varphi(z)=\varphi(y z)=\varphi(z y)=\varphi(z) \varphi(y)=r x .
$$
Thus $r \in Z(S)$.
\end{proof}


\paragraph{Exercise 7.3.37} An ideal $N$ is called nilpotent if $N^{n}$ is the zero ideal for some $n \geq 1$. Prove that the ideal $p \mathbb{Z} / p^{m} \mathbb{Z}$ is a nilpotent ideal in the ring $\mathbb{Z} / p^{m} \mathbb{Z}$.
\begin{proof}
    First we prove a lemma.
Lemma: Let $R$ be a ring, and let $I_1, I_2, J \subseteq R$ be ideals such that $J \subseteq I_1, I_2$. Then $\left(I_1 / J\right)\left(I_2 / J\right)=I_1 I_2 / J$.
Proof: ( $\subseteq$ ) Let
$$
\alpha=\sum\left(x_i+J\right)\left(y_i+J\right) \in\left(I_1 / J\right)\left(I_2 / J\right) .
$$
Then
$$
\alpha=\sum\left(x_i y_i+J\right)=\left(\sum x_i y_i\right)+J \in\left(I_1 I_2\right) / J .
$$
Now let $\alpha=\left(\sum x_i y_i\right)+J \in\left(I_1 I_2\right) / J$. Then
$$
\alpha=\sum\left(x_i+J\right)\left(y_i+J\right) \in\left(I_1 / J\right)\left(I_2 / J\right) .
$$
From this lemma and the lemma to Exercise 7.3.36, it follows by an easy induction that
$$
\left(p \mathbb{Z} / p^m \mathbb{Z}\right)^m=(p \mathbb{Z})^m / p^m \mathbb{Z}=p^m \mathbb{Z} / p^m \mathbb{Z} \cong 0 .
$$
Thus $p \mathbb{Z} / p^m \mathbb{Z}$ is nilpotent in $\mathbb{Z} / p^m \mathbb{Z}$.
\end{proof}



\paragraph{Exercise 7.4.27} Let $R$ be a commutative ring with $1 \neq 0$. Prove that if $a$ is a nilpotent element of $R$ then $1-a b$ is a unit for all $b \in R$.
\begin{proof}
    $\mathfrak{N}(R)$ is an ideal of $R$. Thus for all $b \in R,-a b$ is nilpotent. Hence $1-a b$ is a unit in $R$.
\end{proof}



\paragraph{Exercise 8.1.12} Let $N$ be a positive integer. Let $M$ be an integer relatively prime to $N$ and let $d$ be an integer relatively prime to $\varphi(N)$, where $\varphi$ denotes Euler's $\varphi$-function. Prove that if $M_{1} \equiv M^{d} \pmod N$ then $M \equiv M_{1}^{d^{\prime}} \pmod N$ where $d^{\prime}$ is the inverse of $d \bmod \varphi(N)$: $d d^{\prime} \equiv 1 \pmod {\varphi(N)}$.
\begin{proof}
    Note that there is some $k \in \mathbb{Z}$ such that $M^{d d^{\prime}} \equiv M^{k \varphi(N)+1} \equiv\left(M^{\varphi(N)}\right)^k \cdot M \bmod N$. By Euler's Theorem we have $M^{\varphi(N)} \equiv 1 \bmod N$, so that $M_1^{d^{\prime}} \equiv M \bmod N$.
\end{proof}



\paragraph{Exercise 8.2.4} Let $R$ be an integral domain. Prove that if the following two conditions hold then $R$ is a Principal Ideal Domain: (i) any two nonzero elements $a$ and $b$ in $R$ have a greatest common divisor which can be written in the form $r a+s b$ for some $r, s \in R$, and (ii) if $a_{1}, a_{2}, a_{3}, \ldots$ are nonzero elements of $R$ such that $a_{i+1} \mid a_{i}$ for all $i$, then there is a positive integer $N$ such that $a_{n}$ is a unit times $a_{N}$ for all $n \geq N$.
\begin{proof}
    Let $I \leq R$ be a nonzero ideal and let $I / \sim$ be the set of equivalence classes of elements of $I$ with regards to the relation of being associates. We can equip $I / \sim$ with a partial order with $[x] \leq[y]$ if $y \mid x$. Condition (ii) implies all chains in $I / \sim$ have an upper bound, so By Zorn's lemma $I / \sim$ contains a maximal element, i.e. $I$ contains a class of associated elements which are minimal with respect to divisibility.

Now let $a, b \in I$ be two elements such that $[a]$ and $[b]$ are minimal with respect to divisibility. By condition (i) $a$ and $b$ have a greatest common divisor $d$ which can be expressed as $d=$ $a x+b y$ for some $x, y \in R$. In particular, $d \in I$. Since $a$ and $b$ are minimal with respect to divisibility, we have that $[a]=[b]=[d]$. Therefore $I$ has at least one element $a$ that is minimal with regard to divisibility and all such elements are associate, and we have $I=\langle a\rangle$ and so $I$ is principal. We conclude $R$ is a principal ideal domain.
\end{proof}



\paragraph{Exercise 8.3.4} Prove that if an integer is the sum of two rational squares, then it is the sum of two integer squares.
\begin{proof}
    Let $n=\frac{a^2}{b^2}+\frac{c^2}{d^2}$, or, equivalently, $n(b d)^2=a^2 d^2+c^2 b^2$. From this, we see that $n(b d)^2$ can be written as a sum of two squared integers. Therefore, if $q \equiv 3(\bmod 4)$ and $q^i$ appears in the prime power factorization of $n, i$ must be even. Let $j \in \mathbb{N} \cup\{0\}$ such that $q^j$ divides $b d$. Then $q^{i-2 j}$ divides $n$. But since $i$ is even, $i-2 j$ is even as well. Consequently, $n$ can be written as a sum of two squared integers.
\end{proof}



\paragraph{Exercise 8.3.5a} Let $R=\mathbb{Z}[\sqrt{-n}]$ where $n$ is a squarefree integer greater than 3. Prove that $2, \sqrt{-n}$ and $1+\sqrt{-n}$ are irreducibles in $R$.
\begin{proof}
    Suppose $a=a_1+a_2 \sqrt{-n}, b=b_1+b_2 \sqrt{-n} \in R$ are such that $2=a b$, then $N(a) N(b)=4$. Without loss of generality we can assume $N(a) \leq N(b)$, so $N(a)=1$ or $N(a)=2$. Suppose $N(a)=2$, then $a_1^2+n a_2^2=2$ and since $n>3$ we have $a_2=0$, which implies $a_1^2=2$, a contradiction. So $N(a)=1$ and $a$ is a unit. Therefore 2 is irreducible in $R$.

Suppose now $\sqrt{-n}=a b$, then $N(a) N(b)=n$ and we can assume $N(a)<$ $N(b)$ since $n$ is square free. Suppose $N(a)>1$, then $a_1^2+n a_2^2>1$ and $a_1^2+n a_2^2 \mid n$, so $a_2=0$, and therefore $a_1^2 \mid n$. Since $n$ is squarefree, $a_1=\pm 1$, a contradiction. Therefore $N(a)=1$ and so $a$ is a unit and $\sqrt{-n}$ is irreducible.

Suppose $1+\sqrt{-n}=a b$, then $N(a) N(b)=n+1$ and we can assume $N(a) \leq N(b)$. Suppose $N(a)>1$, then $a_1^2+n a_2^2>1$ and $a_1^2+n a_2^2 \mid n+1$. If $\left|a_2\right| \geq 2$, then since $n>3$ we have a contradiction since $N(a)$ is too large. If $\left|a_2\right|=1$, then $a_1^2+n$ divides $1+n$ and so $a_1=\pm 1$, and in either case $N(a)=n+1$ which contradicts $N(a) \leq N(b)$. If $a_2=0$ then $a_1^2\left(b_1^2+n b_2^2\right)=\left(a_1 b_1\right)^2+n\left(a_1 b_2\right)^2=n+1$. If $\left|a_1 b_2\right| \geq 2$ we have a contradiction. If $\left|a_1 b_2\right|=1$ then $a_1=\pm 1$ which contradicts $N(a)>1$. If $\left|a_1 b_2\right|=0$, then $b_2=0$ and so $a_1 b_1=\sqrt{-n}$, a contradiction. Therefore $N(a)=1$ and so $a$ is a unit and $1+\sqrt{-n}$ is irreducible.
\end{proof}



\paragraph{Exercise 8.3.6a} Prove that the quotient ring $\mathbb{Z}[i] /(1+i)$ is a field of order 2.
\begin{proof}
    Let $a+b i \in \mathbb{Z}[i]$. If $a \equiv b \bmod 2$, then $a+b$ and $b-a$ are even and $(1+i)\left(\frac{a+b}{2}+\frac{b-a}{2} i\right)=a+b i \in\langle 1+i\rangle$. If $a \not \equiv b \bmod 2$ then $a-1+b i \in\langle 1+i\rangle$. Therefore every element of $\mathbb{Z}[i]$ is in either $\langle 1+i\rangle$ or $1+\langle 1+i\rangle$, so $\mathbb{Z}[i] /\langle 1+i\rangle$ is a finite ring of order 2 , which must be a field.
\end{proof}



\paragraph{Exercise 8.3.6b} Let $q \in \mathbb{Z}$ be a prime with $q \equiv 3 \bmod 4$. Prove that the quotient ring $\mathbb{Z}[i] /(q)$ is a field with $q^{2}$ elements.
\begin{proof}
    The division algorithm gives us that every element of $\mathbb{Z}[i] /\langle q\rangle$ is represented by an element $a+b i$ such that $0 \leq a, b<q$. Each such choice is distinct since if $a_1+b_1 i+\langle q\rangle=a_2+b_2 i+\langle q\rangle$, then $\left(a_1-a_2\right)+\left(b_1-b_2\right) i$ is divisible by $q$, so $a_1 \equiv a_2 \bmod q$ and $b_1 \equiv b_2 \bmod q$. So $\mathbb{Z}[i] /\langle q\rangle$ has order $q^2$.

Since $q \equiv 3 \bmod 4, q$ is irreducible, hence prime in $\mathbb{Z}[i]$. Therefore $\langle q\rangle$ is a prime ideal in $\mathbb{Z}[i]$, and so $\mathbb{Z}[i] /\langle q\rangle$ is an integral domain. So $\mathbb{Z}[i] /\langle q\rangle$ is a field.
\end{proof}



\paragraph{Exercise 9.1.6} Prove that $(x, y)$ is not a principal ideal in $\mathbb{Q}[x, y]$.
\begin{proof}
    Suppose, to the contrary, that $(x, y)=p$ for some polynomial $p \in \mathbb{Q}[x, y]$. From $x, y \in$ $(x, y)=(p)$ there are $s, t \in \mathbb{Q}[x, y]$ such that $x=s p$ and $y=t p$.
Then:
$$
\begin{aligned}
& 0=\operatorname{deg}_y(x)=\operatorname{deg}_y(s)+\operatorname{deg}_y(p) \text { so } \\
& 0=\operatorname{deg}_y(p) \\
& 0=\operatorname{deg}_x(y)=\operatorname{deg}_x(s)+\operatorname{deg}_x(p) \text { so } \\
& 0=\operatorname{deg}_x(p) \text { so }
\end{aligned}
$$
From : $\quad 0=\operatorname{deg}_y(p)=\operatorname{deg}_x(p)$ we get $\operatorname{deg}(p)=0$ and $p \in \mathbb{Q}$.
But $p \in(p)=(x, y)$ so $p=a x+b y$ for some $a, b \in \mathbb{Q}[x, y]$
$$
\begin{aligned}
\operatorname{deg}(p) & =\operatorname{deg}(a x+b y) \\
& =\min (\operatorname{deg}(a)+\operatorname{deg}(x), \operatorname{deg}(b)+\operatorname{deg}(y)) \\
& =\min (\operatorname{deg}(a)+1, \operatorname{deg}(b)+1) \geqslant 1
\end{aligned}
$$
which contradicts $\operatorname{deg}(p)=0$.
So we conclude that $(x, y)$ is not principal ideal in $\mathbb{Q}[x, y]$
\end{proof}



\paragraph{Exercise 9.1.10} Prove that the ring $\mathbb{Z}\left[x_{1}, x_{2}, x_{3}, \ldots\right] /\left(x_{1} x_{2}, x_{3} x_{4}, x_{5} x_{6}, \ldots\right)$ contains infinitely many minimal prime ideals.
\begin{proof}
    Let $R=\mathbb{Z}\left[x_1, x_2, \ldots, x_n\right]$ and consider the ideal $K=\left(x_{2 k+1} x_{2 k+2} \mid k \in \mathbb{Z}_{+}\right)$in $R$.
Consider the family of subsets $X=\left\{\left\{x_{2 k+1}, x_{2 k+2}\right\} \mid k \in \mathbb{Z}_{+}\right\}$, and $Y$ the set of choice function on $X$, ie the set of functions $\lambda: \mathbb{Z}_{+} \rightarrow \cup_{\mathbb{Z}_{+}}\left\{x_{2 k+1}, x_{2 k+2}\right\}$ with $\lambda(a) \in$ $\left\{x_{2 a+1}, x_{2 a+2}\right\}$
For each $\lambda \in Y$ we have the ideal $I_\lambda=(\lambda(0), \lambda(1), \ldots)$.
All these ideals are distinct, ie for $\lambda \neq \lambda^{\prime}$ we have $I_\lambda \neq I_{\lambda^{\prime}}$.
We also have that by construction $K \subset I_\lambda$ for all $\lambda \in Y$.
By the Third Isomorphism Theorem
$$
(R / K) /\left(I_\lambda / K\right) \cong R / I_\lambda
$$
Note also that $R / I_\lambda$ is isomorphic to the polynomial ring over $R$ with indeterminates the $x_i$ not in the image of $\lambda$, and since there is a countably infinite number of them we can conclude $R / I_\lambda \cong R$, an integral domain. Therefore $I_\lambda / K$ is a prime ideal of $R / K$

We prove now that $I_\lambda / K$ is a minimal prime ideal. Let $J / K \subseteq I_\lambda / K$ be a prime ideal. For each pair $\left(x_{2 k+1}, x_{2 k+2}\right)$ we have that $x_{2 k+1} x_{2 k+2} \in K$ so $x_{2 k+1} x_{2 k+2} \bmod K \in J / K$ so $J$ must contain one of the elements in $\left\{x_{2 k+1}, x_{2 k+2}\right\}$. But since $J / K \subseteq I_\lambda / K$ it must be $\lambda(k)$ for all $k \in \mathbb{Z}_{+}$. Therefore $J / K=I_\lambda / K$
\end{proof}



\paragraph{Exercise 9.3.2} Prove that if $f(x)$ and $g(x)$ are polynomials with rational coefficients whose product $f(x) g(x)$ has integer coefficients, then the product of any coefficient of $g(x)$ with any coefficient of $f(x)$ is an integer.
\begin{proof}
    Let $f(x), g(x) \in \mathbb{Q}[x]$ be such that $f(x) g(x) \in \mathbb{Z}[x]$.
By Gauss' Lemma there exists $r, s \in \mathbb{Q}$ such that $r f(x), s g(x) \in \mathbb{Z}[x]$, and $(r f(x))(s g(x))=r s f(x) g(x)=f(x) g(x)$. From this last relation we can conclude that $s=r^{-1}$.

Therefore for any coefficient $f_i$ of $f(x)$ and $g_j$ of $g(x)$ we have that $r f_i, r^{-1} g_j \in$ $\mathbb{Z}$ and by multiplicative closure and commutativity of $\mathbb{Z}$ we have that $r f_i r^{-1} g_j=$ $f_i g_j \in \mathbb{Z}$
\end{proof}



\paragraph{Exercise 9.4.2a} Prove that $x^4-4x^3+6$ is irreducible in $\mathbb{Z}[x]$.
\begin{proof}
$$
x^4-4 x^3+6
$$
The polynomial is irreducible by Eisenstein's Criterion since the prime $2$ doesn't divide the leading coefficient 2 divide coefficients of the low order term $-4,0,0$ but 6 is not divided by the square of 2.
\end{proof}



\paragraph{Exercise 9.4.2b} Prove that $x^6+30x^5-15x^3 + 6x-120$ is irreducible in $\mathbb{Z}[x]$.
\begin{proof}
    $$
x^6+30 x^5-15 x^3+6 x-120
$$
The coefficients of the low order.: $30,-15,0,6,-120$
They are divisible by the prime 3 , but $3^2=9$ doesn't divide $-120$. So this polynomial is irreducible over $\mathbb{Z}$. 
\end{proof}



\paragraph{Exercise 9.4.2c} Prove that $x^4+4x^3+6x^2+2x+1$ is irreducible in $\mathbb{Z}[x]$.
\begin{proof}
$$
p(x)=x^4+6 x^3+4 x^2+2 x+1
$$
We calculate $p(x-1)$
$$
\begin{aligned}
(x-1)^4 & =x^4-4 x^3+6 x^2-4 x+1 \\
6(x-1)^3 & =6 x^3-18 x^2+18 x-6 \\
4(x-1)^2 & =4 x^2-8 x+4 \\
2(x-1) & =2 x-2 \\
1 & =1
\end{aligned}
$$
$$
\begin{aligned}
& p(x-1)=(x-1)^4+6(x-1)^3+4(x-1)^2+2(x-1)+1=x^4+2 x^3-8 x^2+ \\
& 8 x-2 \\
& q(x)=x^4+2 x^3-8 x^2+8 x-2
\end{aligned}
$$
$q(x)$ is irreducible by Eisenstein's Criterion since the prime $\$ 2 \$$ divides the lower coefficient but $\$ 2^{\wedge} 2 \$$ doesnt divide constant $-2$. Any factorization of $p(x)$ would provide a factor of $p(x)(x-1)$
Since:
$$
\begin{aligned}
& p(x)=a(x) b(x) \\
& q(x)=p(x)(x-1)=a(x-1) b(x-1)
\end{aligned}
$$
We get a contradiction with the irreducibility of $p(x-1)$, so $p(x)$ is irreducible in $Z[x]$
\end{proof}



\paragraph{Exercise 9.4.2d} Prove that $\frac{(x+2)^p-2^p}{x}$, where $p$ is an odd prime, is irreducible in $\mathbb{Z}[x]$.
\begin{proof}
$\frac{(x+2)^p-2^p}{x} \quad \quad p$ is on add pprime $Z[x]$
$$
\frac{(x+2)^p-2^p}{x} \quad \text { as a polynomial we expand }(x+2)^p
$$
$2^p$ cancels with $-2^p$, every remaining term has $x$ as $a$ factor
$$
\begin{aligned}
& x^{p-1}+2\left(\begin{array}{l}
p \\
1
\end{array}\right) x^{p-2}+2^2\left(\begin{array}{l}
p \\
2
\end{array}\right) x^{p-3}+\ldots+2^{p-1}\left(\begin{array}{c}
p \\
p-1
\end{array}\right) \\
& 2^k\left(\begin{array}{l}
p \\
k
\end{array}\right) x^{p-k-1}=2^k \cdot p \cdot(p-1) \ldots(p-k-1), \quad 0<k<p
\end{aligned}
$$
Every lower order coef. has $p$ as a factor but doesnt have $\$ \mathrm{p}^{\wedge} 2 \$$ as a fuction so the polynomial is irreducible by Eisenstein's Criterion.
\end{proof}



\paragraph{Exercise 9.4.9} Prove that the polynomial $x^{2}-\sqrt{2}$ is irreducible over $\mathbb{Z}[\sqrt{2}]$. You may assume that $\mathbb{Z}[\sqrt{2}]$ is a U.F.D.
\begin{proof}
$Z[\sqrt{2}]$ is an Euclidean domain, and so a unique factorization domain.
We have to prove $p(x)=x^2-\sqrt{2}$ irreducible.
Suppose to the contrary.
if $p(x)$ is reducible then it must have root.
Let $a+b \sqrt{2}$ be a root of $x^2-\sqrt{2}$.
Now we have
$$
a^2+2 b^2+2 a b \sqrt{2}=\sqrt{2}
$$
By comparing the coefficients we get $2 a b=1$ for some pair of integers $a$ and $b$, a contradiction.
So $p(x)$ is irredicible over $Z[\sqrt{2}]$.
\end{proof}



\paragraph{Exercise 9.4.11} Prove that $x^2+y^2-1$ is irreducible in $\mathbb{Q}[x,y]$.
\begin{proof}
$$
p(x)=x^2+y^2-1 \in Q[y][x] \cong Q[y, x]
$$
We have that $y+1 \in Q[y]$ is prime and $Q[y]$ is an UFD, since $p(x)=x^2+y^2-1=x^2+$ $(y+1)(y-1)$ by the Eisenstein's Criterion $x^2+y^2-1$ is irreducibile in $Q[x, y]$.
\end{proof}



\paragraph{Exercise 11.1.13} Prove that as vector spaces over $\mathbb{Q}, \mathbb{R}^n \cong \mathbb{R}$, for all $n \in \mathbb{Z}^{+}$.
\begin{proof}    
Since $B$ is a basis of $V$, every element of $V$ can be written uniquely as a finite linear combination of elements of $B$. Let $X$ be the set of all such finite linear combinations. Then $X$ has the same cardinality as $V$, since the map from $X$ to $V$ that takes each linear combination to the corresponding element of $V$ is a bijection.

We will show that $X$ has the same cardinality as $B$. Since $B$ is countable and $X$ is a union of countable sets, it suffices to show that each set $X_n$, consisting of all finite linear combinations of $n$ elements of $B$, is countable.

Let $P_n(X)$ be the set of all subsets of $X$ with cardinality $n$. Then we have $X_n \subseteq P_n(B)$. Since $B$ is countable, we have $\mathrm{card}(P_n(B)) \leq \mathrm{card}(B^n) = \mathrm{card}(B)$, where $B^n$ is the Cartesian product of $n$ copies of $B$.

Thus, we have $\mathrm{card}(X_n) \leq \mathrm{card}(P_n(B)) \leq \mathrm{card}(B)$, so $X_n$ is countable. It follows that $X$ is countable, and hence has the same cardinality as $B$.

Therefore, we have shown that the cardinality of $V$ is equal to the cardinality of $B$. Since $F$ is countable, it follows that the cardinality of $V$ is countable as well.

Now let $Q$ be a countable field, and let $R$ be a vector space over $Q$. Let $n$ be a positive integer. Then any basis of $R^n$ over $Q$ has the same cardinality as $R^n$, which is countable. Since $R$ is a direct sum of $n$ copies of $R^n$, it follows that any basis of $R$ over $Q$ has the same cardinality as $R$. Hence, the cardinality of $R$ is countable.

Finally, since $R$ is a countable vector space and $Q$ is a countable field, it follows that $R$ and $Q^{\oplus \mathrm{card}(R)}$ are isomorphic as additive abelian groups. Therefore, we have $R \cong_Q Q^{\oplus \mathrm{card}(R)}$, and in particular $R \cong_Q R^n$ for any positive integer $n$.
\end{proof}




\end{document}
